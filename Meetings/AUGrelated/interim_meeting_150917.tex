\documentclass[10pt, compress]{beamer}
\usetheme[conference=Topic 21,venue=Remote, date=15/09/2017, titleprogressbar, logo=RFX-logo]{Eurof}
\usepackage{listings,amsmath,multimedia, amssymb}
\usepackage{../beamerclass/tangocolors}
\usepackage{../beamerclass/rfxcolor}
% for drawing
\usepackage{pgf}
\usepackage{tikz}
\usetikzlibrary{arrows,shapes,backgrounds}
\usepackage{../beamerclass/onimage}
\usepackage[export]{adjustbox}
\usepackage{bm}
% for font
\usepackage[absolute,overlay]{textpos}
  \setlength{\TPHorizModule}{1mm}
  \setlength{\TPVertModule}{1mm}

\usepackage[style=nature,citestyle=authoryear-comp,defernumbers=true,maxnames=2,firstinits=true,
uniquename=init,backend=bibtex8,arxiv=abs,mcite]{biblatex}
\bibliography{biblio}
\renewcommand*{\bibfont}{\footnotesize}
\renewcommand*{\citesetup}{\footnotesize}
\usepackage[export]{adjustbox}
\makeatother
\mode<presentation>
\makeatletter
% add a macro that saves its argument
\newcommand{\footlineextra}[1]{\gdef\insertfootlineextra{#1}}
\newbox\footlineextrabox
% for reducing font on a single slide
\newcommand\Fontvi{\fontsize{8}{7.2}\selectfont}
\title{{\small Topic 21: AUG experiment analsysis meeting }}
\date{15 September 2017}
\author[Topic 21 Scientific Team]{N. Vianello and V. Naulin for the
  Topic 21 Scientific Team}
\begin{document}
\tikzstyle{every picture}+=[remember picture]
\maketitle
\begin{frame}{Summary of the campaigns}
  \begin{itemize}[<+-|alert@+>]
  \item L-Mode experiment, CW 17.
    \begin{enumerate}
      \item<1-> Performed similar density ramps in an I$_p$ scan at constant q$_{95}$
      \item<1-> Performed similar density ramps in an I$_p$ scan at constant B$_t$ 
    \end{enumerate}    
    \item H-Mode experiment, CW 21. 
      \begin{enumerate}
    \item<2> Compare divertor/midplane fueling effect on filamentary
      transport and profiles without cryo-pumps
    \item<2> Compare profiles with the same fueling with/without cryopums
    \item<2> Determine an H-Mode with the cryopumps matching similar
      divertor pressure and SOL profiles
    \end{enumerate}
  \end{itemize}    
\end{frame}

\begin{frame}{L-Mode analysis: I$_p$ scan at constant q$_95$}
\Fontvi
  \vspace{-1cm}
\begin{columns}
  \begin{column}{0.65\textwidth}
    \centering{\includegraphics<1>[height=.8\textheight]{../../Experiments/AUG/analysis/pdfbox/EquilibraLparallelConstantQ95}}
    \centering{\includegraphics<2>[width=\textwidth]{../../Experiments/AUG/analysis/pdfbox/GeneralIpScanConstantq95}}
    \centering{\includegraphics<3>[width=\textwidth]{../../Experiments/AUG/analysis/pdfbox/IpConstantQ95_Profiles_UsDiv}}

  \end{column}
  \begin{column}{0.35\textwidth}
    \begin{itemize}
      \item<1|only@1> We matched correctly the shape and the L$_{\parallel}$
        here shown from outer divertor plate up to X-point 
      \item<2|only@2> The scan was performed with similar puffing rate (0.8-1
        MA) whereas we reduced it at lower current to avoid early disruption
      \item<2|only@2> We have comparable edge density and divertor neutral
        pressure 
      \item<3> At comparable edge density Upstream profiles are
        different with the tendency to develop shoulder easier at
        lower current
      \end{itemize}
    \end{column}
\end{columns}
\end{frame}


% \begin{frame}{Compare Similar fueling with/without cryopumps}
% \Fontvi
%   \vspace{-1cm}
% \begin{columns}
%   \begin{column}{0.65\textwidth}
%     \centering{\includegraphics<1>[width=\textwidth]{../../Experiments/AUG/analysis/pdfbox/CompareShot34276_34278}}
%     \centering{\includegraphics<2>[width=\textwidth]{../../Experiments/AUG/analysis/pdfbox/EvolutionEdgeProfiles_34276_34278}}
%   \end{column}
%   \begin{column}{0.35\textwidth}
%     \begin{itemize}
%       \item<1-> Same fueling but with cryo-pumps. Clearly different in
%         terms of Edge density and Divertor pressure
%       \item<2-> Also with this amount of fueling any instance of SOL
%         saturation observed
%     \end{itemize}
  
%   \end{column}
% \end{columns}
% \end{frame}

% \begin{frame}{Matching scenarios with cryo-pumps}
% \Fontvi
%   \vspace{-1cm}
% \begin{columns}
%   \begin{column}{0.65\textwidth}
%     \centering{\includegraphics<1>[width=\textwidth]{../../Experiments/AUG/analysis/pdfbox/CompareShot34276_34281}}
%     \centering{\includegraphics<2>[width=\textwidth]{../../Experiments/AUG/analysis/pdfbox/EvolutionEdgeProfiles_34276_34280}}
%   \end{column}
%   \begin{column}{0.35\textwidth}
%     \begin{itemize}
%       \item<1-> To match similar edge density and divertor pressure
%         and to reach the same level of detachment we increase the
%         fueling by almost a factor of 3, increasing also the rate. In
%         addition to that we also increase substantially the N puffing
%       \item<2-> Li-beam profile not yet produced for the same
%         shots. With a lower level of N (no detachment observed) the
%         SOL profiles does not flatten as in the case with the cryo-pumps
%     \end{itemize}
%   \end{column}
% \end{columns}
% \end{frame}

% \begin{frame}{Work in progress}
%   \begin{itemize}
%     \item Confirmed SXR spikes correlated with the start of ELMs,
%       strongly suggesting electron accelleration during ELM filament
%       eruption
%     \item Evaluation progressing in terms of fluctuation analysis from
%       MEM, Reflectometry, Li-Beam
%     \item GPI data available in different density scenarios and also
%       during L-H transition
%   \end{itemize}
% \end{frame}
\end{document}

