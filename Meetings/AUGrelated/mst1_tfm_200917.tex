\documentclass[10pt, compress]{beamer}
\usetheme[conference=MST1-TFM,venue=Topic-21, date=20/09/2017, titleprogressbar, logo=RFX-logo]{Eurof}
\usepackage{listings,amsmath,multimedia, amssymb}
\usepackage{../beamerclass/tangocolors}
\usepackage{../beamerclass/rfxcolor}
% for drawing
\usepackage{pgf}
\usepackage{tikz}
\usetikzlibrary{arrows,shapes,backgrounds}
\usepackage{../beamerclass/onimage}
\usepackage[export]{adjustbox}
\usepackage{bm}
% for font
\usepackage[absolute,overlay]{textpos}
  \setlength{\TPHorizModule}{1mm}
  \setlength{\TPVertModule}{1mm}

\usepackage[style=nature,citestyle=authoryear-comp,defernumbers=true,maxnames=2,firstinits=true,
uniquename=init,backend=bibtex8,arxiv=abs,mcite]{biblatex}
\bibliography{biblio}
\renewcommand*{\bibfont}{\footnotesize}
\renewcommand*{\citesetup}{\footnotesize}
\usepackage[export]{adjustbox}
\makeatother
\mode<presentation>
\makeatletter
% add a macro that saves its argument
\newcommand{\footlineextra}[1]{\gdef\insertfootlineextra{#1}}
\newbox\footlineextrabox
% for reducing font on a single slide
\newcommand\Fontvi{\fontsize{8}{7.2}\selectfont}
\title{{\small Topic 21: Filamentary transport in high-power H-mode conditions and in no/small-ELM regimes to predict heat and particle loads on PFCs for future devices }}
\date{20 September 2017}
\author[N. Vianello]{N. Vianello and V. Naulin for the
  Topic 21 Scientific Team}
\begin{document}
\tikzstyle{every picture}+=[remember picture]
\maketitle
\begin{frame}{Summary of the campaigns}
  \begin{itemize}
  \item<1-> L-Mode experiment, Calendar week
    17. It will constitute baseline scenarios to be replicated on
    the other MST1 devices. \textcolor{ta3scarletred}{Operation without cryopumps}
    \begin{enumerate}
      \item<1-> \color<2->{ta3chameleon}{Performed similar density ramps in an I$_p$ scan at constant q$_{95}$}
      \item<1-> \color<2->{ta3chameleon}{Performed similar density ramps in an I$_p$ scan at
        constant B$_t$}
      \item<1-> \color<2->{ta3chameleon}{Prepare scenarios for H-Mode operation}
      \end{enumerate}
    \item<3-> H-Mode experiment, CW 21. 
      \begin{enumerate}
      \item<3-> \color<4->{ta3scarletred}{Compare divertor/midplane fueling effect on filamentary
        transport and profiles without cryo-pumps}
      \item<3-> \color<4->{ta3chameleon}{Compare profiles with the same fueling with/without cryopums}
      \item<3-> \color<4->{orange}{Determine an H-Mode with the cryopumps matching similar
        divertor pressure and SOL profiles}
      \end{enumerate}
    \end{itemize}    
  \end{frame}


\begin{frame}{L-Mode analysis: I$_p$ scan at constant q$_{95}$}
\Fontvi
  \vspace{-1cm}
\begin{columns}
  \begin{column}{0.65\textwidth}
    \centering{\includegraphics<1>[height=.8\textheight]{../../Experiments/AUG/analysis/pdfbox/EquilibraLparallelConstantQ95}}
    \centering{\includegraphics<2>[width=\textwidth]{../../Experiments/AUG/analysis/pdfbox/GeneralIpScanConstantq95}}
    \centering{\includegraphics<3>[width=\textwidth]{../../Experiments/AUG/analysis/pdfbox/IpConstantQ95_Profiles_UsDiv}}
    \centering{\includegraphics<4>[width=\textwidth]{../../Experiments/AUG/analysis/pdfbox/PdfStructureCurrentScan_ConstantQ95}}

  \end{column}
  \begin{column}{0.35\textwidth}
    \begin{itemize}
      \item<1|only@1> All the shots were performed in the so-called
        Edge Optmized Configuration (EOC) shape
      \item<1|only@1> We matched correctly the shape and the L$_{\parallel}$
        here shown from outer divertor plate up to X-point 
      \item<2|only@2> The scan was performed with similar puffing rate (0.8-1
        MA) whereas we reduced it at lower current to avoid early
        disruption
      \item<2|only@2> The total power (Ohmic plus NBI) was kept
        constant throughout the scan
      \item<2|only@2> We have comparable edge density, divertor neutral
        pressure and divertor temperature
      \item<3|only@3> At comparable edge density Upstream profiles are
        different with the tendency to develop shoulder easier at
        lower current
      \item<3|only@3> Still need to provide detail evolution of edge
        profiles as a function
        of $\Lambda_{div}$ which are different at different current
        although same L$_{\parallel}$
      \item<4|only@4> No sensible difference of PDFs of J$_{sat}$ at
        different current even though obtained at different values of $\Lambda_{div}$. 
      \item<4|only@4> Autocorrelation time $\tau_{ac}$ increases with
        $\Lambda_{div}$ without sensible difference among the current
      \end{itemize}
    \end{column}
\end{columns}
\end{frame}

\begin{frame}{L-Mode analysis: I$_p$ scan at constant B$_t$}
\Fontvi
  \vspace{-1cm}
\begin{columns}
  \begin{column}{0.65\textwidth}
    \centering{\includegraphics<1>[height=.8\textheight]{../../Experiments/AUG/analysis/pdfbox/EquilibraLparallelConstantBt}}
    \centering{\includegraphics<2>[width=\textwidth]{../../Experiments/AUG/analysis/pdfbox/GeneralIpScanConstantBt}}
    \centering{\includegraphics<3>[width=\textwidth]{../../Experiments/AUG/analysis/pdfbox/IpConstantBt_Profiles_UsDiv}}
    \centering{\includegraphics<4>[width=\textwidth]{../../Experiments/AUG/analysis/pdfbox/PdfStructureCurrentScan_ConstantBt}}
  \end{column}
  \begin{column}{0.35\textwidth}
    \begin{itemize}
      \item<1|only@1> We matched correctly the shape the parallel
        connection length L$_{\parallel}$ is modified consistently
      \item<2|only@2> The scan was performed with similar puffing rate (0.8-1
        MA) whereas we reduced it at lower current to avoid early disruption
      \item<2|only@2> We have comparable edge density and divertor neutral
        pressure even though pressure increase earlier at higher current
      \item<3|only@3> At comparable edge density Upstream profiles are
        different with the tendency to develop shoulder easier at
        lower current
      \item<3|only@3> Still need to provide detail evolution of edge
        profiles as a function
        of $\Lambda_{div}$ which are different at different current
        although same L$_{\parallel}$
      \item<4|only@4> No sensible difference of PDFs of J$_{sat}$ at
        different current even though obtained at different values of $\Lambda_{div}$. 
      \item<4|only@4> Remarkable difference in the shape of typical
        structure,  even though $\tau_{ac}$ follow usual trend apart
        from very last points. \alert{To be double-checked}
      \end{itemize}
    \end{column}
\end{columns}
\end{frame}

\begin{frame}{H-Mode investigation: puffing location}
\Fontvi
  \vspace{-1cm}
\begin{columns}
  \begin{column}{0.65\textwidth}
    \centering{\includegraphics<1>[height=0.85\textheight]{../../Experiments/AUG/analysis/pdfbox/PuffingLocation}}
    \centering{\includegraphics<2>[width=\textwidth]{../../Experiments/AUG/analysis/pdfbox/CompareShot34276_34277}}
    \only<3>{
      \begin{tikzonimage}[width=\textwidth]{../../Experiments/AUG/analysis/pdfbox/EvolutionEdgeProfiles_34276_34277}
        \draw [->, ultra thick, white] (0.2, 0.45) -- (0.3, 0.55);
        \draw [->, ultra thick, white] (0.62, 0.45) -- (0.72, 0.55);
      \end{tikzonimage}   
    }
  \end{column}
  \begin{column}{0.35\textwidth}
    \begin{itemize}
      \item<1-> Similar puff from Lower and Upper divertor valves
        (\alert{we asked for divertor/midplane valves})
      \item<2-> Discharge with a total amount 6.5 heating power
        with equivalent behavior also in the lower divertor
        independently from the puffing location
      \item<3-> Edge density profiles from Li-Beam evolution are
        pretty similar
      \item<3|alert@3> Similar behavior observed from RIC Antenna 4
        for the available shot
    \end{itemize}
  \end{column}
\end{columns}
\end{frame}


\begin{frame}{Compare Similar fueling with/without cryopumps}
\Fontvi
  \vspace{-1cm}
\begin{columns}
  \begin{column}{0.65\textwidth}
    \centering{\includegraphics<1>[width=\textwidth]{../../Experiments/AUG/analysis/pdfbox/CompareShot34276_34278}}
    \centering{\includegraphics<2>[height=.65\textheight]{../../Experiments/AUG/analysis/pdfbox/PuffingIpolsola34276_34278}}
    \only<3>{
    \begin{tikzonimage}[width=\textwidth]{../../Experiments/AUG/analysis/pdfbox/EvolutionEdgeProfiles_34276_34278}
      \draw [->, ultra thick, white] (0.2, 0.45) -- (0.3, 0.55);
      \draw [->, ultra thick, dashed, red] (0.62, 0.45) -- (0.72, 0.55);
    \end{tikzonimage}}
  \centering{\includegraphics<4>[height=.85\textheight]{../../Experiments/AUG/analysis/pdfbox/PdfStructureHmodeCryoOnOff}}
  \end{column}
  \begin{column}{0.35\textwidth}
    \begin{itemize}
    \item<1-> Same fueling but with cryo-pumps
      \item<1-> H-5 density is different and remain constant, both
        divertor and midplane pressure are reduced (to 1/3
        approximately) no sign of detachment
      \item<2-> Different ELMy regimes reached with reduced size and
        increased frequency without the crypumps
      \item<3-> Also with this amount of fueling no instance of SOL
        saturation observed as confirmed by Li-Beam and by RIC (Antenna 4)
      \item<4> Comparison of fluctuations during the plunge at higher
        density (\alert{ELM included}) reveal differences in
        fluctuations. \alert{ELM resolved measurement to be properly done}  
        
    \end{itemize}
  \end{column}
\end{columns}
\end{frame}

\begin{frame}{Matching scenarios with cryo-pumps}
\Fontvi
  \vspace{-1cm}
\begin{columns}
  \begin{column}{0.65\textwidth}
    \centering{\includegraphics<1>[width=\textwidth]{../../Experiments/AUG/analysis/pdfbox/CompareShot34276_34281}}
    \centering{\includegraphics<2>[width=\textwidth]{../../Experiments/AUG/analysis/pdfbox/PuffingIpolsola34276_34281}}
    \centering{\includegraphics<3>[width=\textwidth]{../../Experiments/AUG/analysis/pdfbox/EvolutionEdgeProfiles_34276_34281}}
    4\centering{\includegraphics<4>[height=0.9\textheight]{../../Experiments/AUG/analysis/pdfbox/PdfStructureHmodeCryoOnOffMatch}}
  \end{column}
  \begin{column}{0.35\textwidth}
    \begin{itemize}
      \item<1-> To match similar edge density and divertor pressure
        and to reach the same level of detachment we increase the
        fueling by almost a factor of 3, increasing also the rate. In
        addition to that we also increase substantially the N
        puffing. \alert{Degraded H-Mode reached earlier in density
          without the cryopumps}
      \item<2-> Similar ELMy behavior obtained during the density ramp  
      \item<3-> Without cryopumps evidence of stronger SOL profile
        flattening from Li-Be. \alert{To be compared with appropriate
          variation of $\Lambda_{div}$}
       \item<4> The shape of the PDF is rather similar with closer
         values of autocorrelation time at similar values of divertor
         collisionality. \alert{Not yet ELM resolved}
    \end{itemize}
  \end{column}
\end{columns}
\end{frame}

\begin{frame}{What is missing}
  \Fontvi
  The analysis is still in progress and we are not ready yet to draw
  final conclusions. Among the future on going analysis we can list:
  \begin{enumerate}
    \item RIC profiles with possible poloidal variation among the
      antennas in particular in H-Mode. \textcolor{red}{To be done}
    \item Li-Be ELM resolved profile variation with $\Lambda_{div}$
      \textcolor{red}{To be done}
    \item Li-Be fluctuations \textcolor{red}{To be done}
    \item Structures size and velocities in L-Mode
      \textcolor{orange}{In progress}
    \item Radiation front movement \textcolor{orange}{In progress}
    \item Evolution of divertor profiles from Spectroscopy
      \textcolor{orange}{In progres}
    \item Neutrals analysis from D$_{\alpha}$ camera combined with
      KN1D code for evolution of neutrals
      \textcolor{orange}{In progress}
    \item Fast-camera \textcolor{red}{To be done}
  \end{enumerate}
\end{frame}

\end{document}

