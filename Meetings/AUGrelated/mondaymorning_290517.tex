\documentclass[10pt, compress]{beamer}
\usetheme[conference=AUG Monday,venue=Remote, date=29/05/2017, titleprogressbar, logo=RFX-logo]{Eurof}
\usepackage{listings,amsmath,multimedia, amssymb}
\usepackage{../beamerclass/tangocolors}
\usepackage{../beamerclass/rfxcolor}
% for drawing
\usepackage{pgf}
\usepackage{tikz}
\usetikzlibrary{arrows,shapes,backgrounds}
\usepackage{../beamerclass/onimage}
\usepackage[export]{adjustbox}
\usepackage{bm}
% for font
\usepackage[absolute,overlay]{textpos}
  \setlength{\TPHorizModule}{1mm}
  \setlength{\TPVertModule}{1mm}

\usepackage[style=nature,citestyle=authoryear-comp,defernumbers=true,maxnames=2,firstinits=true,
uniquename=init,backend=bibtex8,arxiv=abs,mcite]{biblatex}
\bibliography{biblio}
\renewcommand*{\bibfont}{\footnotesize}
\renewcommand*{\citesetup}{\footnotesize}
\usepackage[export]{adjustbox}
\makeatother
\mode<presentation>
\makeatletter
% add a macro that saves its argument
\newcommand{\footlineextra}[1]{\gdef\insertfootlineextra{#1}}
\newbox\footlineextrabox
% for reducing font on a single slide
\newcommand\Fontvi{\fontsize{8}{7.2}\selectfont}
\title{{\small Topic 21: Filamentary transport in high-power H-mode conditions and in no/small-ELM regimes to predict heat and particle loads on PFCs for future devices }}
\date{29 May 2017}
\author[Topic 21 Scientific Team]{N. Vianello for the Topic 21 Scientific Team}
\begin{document}
\tikzstyle{every picture}+=[remember picture]
\maketitle
\begin{frame}{Scientific team}
  \begin{center}
N. Vianello, D. Carralero, Z. Wei, J. Madsen, K. McClements,
M. Agostini, M.Spolaore, D. Aguiam, E. Wolfrum, J. Vicente,
L. Florian, E. Seliunin, J. Galdon-Quiroga, C. Ionita, S. Costea ...
  \end{center}
\end{frame}
\begin{frame}{Objective of Week 21 campaign}
  \begin{itemize}[<+-|alert@+>]
    \item Compare divertor/midplane fueling effect on filamentary
      transport and profiles without cryo-pumps
    \item Compare profiles with the same fueling with/without cryopums
    \item Determine an H-Mode with the cryopumps matching similar
      divertor pressure and SOL profiles
  \end{itemize}    
\end{frame}
\begin{frame}{Compare divertor/midplane fueling}
\Fontvi
  \vspace{-1cm}
\begin{columns}
  \begin{column}{0.65\textwidth}
    \centering{\includegraphics<1>[width=\textwidth]{../../Experiments/AUG/analysis/pdfbox/CompareShot34276_34277}}
    \centering{\includegraphics<2>[width=\textwidth]{../../Experiments/AUG/analysis/pdfbox/EvolutionEdgeProfiles_34276_34277}}
  \end{column}
  \begin{column}{0.35\textwidth}
    \begin{itemize}
      \item Similar puff from the divertor or from the midplane
        without Cryopumps. The
        shots are pretty similar also in terms of Divertor pressure
      \item<2-> Edge density profiles from Li-Beam evolution are
        pretty similar
    \end{itemize}
  
  \end{column}
\end{columns}
\end{frame}


\begin{frame}{Compare Similar fueling with/without cryopumps}
\Fontvi
  \vspace{-1cm}
\begin{columns}
  \begin{column}{0.65\textwidth}
    \centering{\includegraphics<1>[width=\textwidth]{../../Experiments/AUG/analysis/pdfbox/CompareShot34276_34278}}
    \centering{\includegraphics<2>[width=\textwidth]{../../Experiments/AUG/analysis/pdfbox/EvolutionEdgeProfiles_34276_34278}}
  \end{column}
  \begin{column}{0.35\textwidth}
    \begin{itemize}
      \item<1-> Same fueling but with cryo-pumps. Clearly different in
        terms of Edge density and Divertor pressure
      \item<2-> Also with this amount of fueling any instance of SOL
        saturation observed
    \end{itemize}
  
  \end{column}
\end{columns}
\end{frame}

\begin{frame}{Matching scenarios with cryo-pumps}
\Fontvi
  \vspace{-1cm}
\begin{columns}
  \begin{column}{0.65\textwidth}
    \centering{\includegraphics<1>[width=\textwidth]{../../Experiments/AUG/analysis/pdfbox/CompareShot34276_34281}}
    \centering{\includegraphics<2>[width=\textwidth]{../../Experiments/AUG/analysis/pdfbox/EvolutionEdgeProfiles_34276_34280}}
  \end{column}
  \begin{column}{0.35\textwidth}
    \begin{itemize}
      \item<1-> To match similar edge density and divertor pressure
        and to reach the same level of detachment we increase the
        fueling by almost a factor of 3, increasing also the rate. In
        addition to that we also increase substantially the N puffing
      \item<2-> Li-beam profile not yet produced for the same
        shots. With a lower level of N (no detachment observed) the
        SOL profiles does not flatten as in the case with the cryo-pumps
    \end{itemize}
  \end{column}
\end{columns}
\end{frame}

\begin{frame}{Work in progress}
  \begin{itemize}
    \item Confirmed SXR spikes correlated with the start of ELMs,
      strongly suggesting electron accelleration during ELM filament
      eruption
    \item Evaluation progressing in terms of fluctuation analysis from
      MEM, Reflectometry, Li-Beam
    \item GPI data available in different density scenarios and also
      during L-H transition
  \end{itemize}
\end{frame}
\end{document}

