\documentclass[10pt, compress]{beamer}
\usetheme[conference=KoM,venue=Remote, date=03/13/2017, titleprogressbar, logo=RFX-logo]{Eurof}
\usepackage{listings,amsmath,multimedia, amssymb}
\usepackage{../beamerclass/tangocolors}
\usepackage{../beamerclass/rfxcolor}
% for drawing
\usepackage{pgf}
\usepackage{tikz}
\usetikzlibrary{arrows,shapes,backgrounds}
\usepackage{../beamerclass/onimage}
\usepackage[export]{adjustbox}
\usepackage{bm}
% for font
\usepackage[absolute,overlay]{textpos}
  \setlength{\TPHorizModule}{1mm}
  \setlength{\TPVertModule}{1mm}

\usepackage[style=nature,citestyle=authoryear-comp,defernumbers=true,maxnames=2,firstinits=true,
uniquename=init,backend=bibtex8,arxiv=abs,mcite]{biblatex}
\bibliography{biblio}
\renewcommand*{\bibfont}{\footnotesize}
\renewcommand*{\citesetup}{\footnotesize}
\usepackage[export]{adjustbox}
\makeatother
\mode<presentation>
\makeatletter
% add a macro that saves its argument
\newcommand{\footlineextra}[1]{\gdef\insertfootlineextra{#1}}
\newbox\footlineextrabox
% for reducing font on a single slide
\newcommand\Fontvi{\fontsize{8}{7.2}\selectfont}
\title{Topic 21: AUG experiment KoM}
\date{03 March 2017}
\author[Topic 21. J. Madsen and N.Vianello]{J. Madsen and N. Vianello}
\begin{document}
\tikzstyle{every picture}+=[remember picture]
\maketitle

\begin{frame}{AUG experimental plan}
\vspace{-1cm}
\only<1>{
  \begin{itemize}
  \item For Topic-21 on AUG 14 shots forseen, split into two blocks of
    7 shot each on week 15 and week 17 respectively
  \item The proposed experimental plan address some of the proposed
experimental plan for both the \textcolor{ta3chameleon}{L-Mode} and
\textcolor{ta3scarletred}{H-Mode} part.
\end{itemize}
}
\only<2>{
We will conclude the \textcolor{ta3chameleon}{L-Mode}
part in Week 15 where we also would like to include part of the \textcolor{ta3scarletred}{H-Mode}
scenario development. This would give us an additional week between
the experiments to address possible issues in particular for the
H-mode part\\
\textbf{Week 15} \\
We choose are reference shot \# 30269 (I$_p$ = 0.8 MA, B$_{\phi}$ = 2.5T, q$_{95}\approx 4.5$)
  \begin{enumerate}
    \item \textcolor{ta3chameleon}{Shot at I$_p$=0.6 MA
      keeping the same toroidal field $B_{\phi}$ = 2.5T as the reference adjust the fueling
      rate (\emph{Current scan with modification of $q_{95}$})}
  \item \textcolor{ta3chameleon}{Shot at I$_p$ = 0.6MA
      reduce the toroidal field B$_{\phi}\approx $2T in order to match q$_{95}\approx 4.5$, 
      (\emph{Current scan at fixed $q_{95}$}). This would accommodate
      the diagnostic request of D. Aguiam}
    \item \textcolor{ta3chameleon}{Density ramp with I$_p$ = 1MA and
        B$_{\phi}$ = 2.5T}
    \item \textcolor{ta3chameleon}{Density ramp with I$_{p}$ = 1MA,
        increasing the toroidal field in order to match q$_{95}\approx
        4.5$}
    \end{enumerate}
  }
  \only<3>{
  \textbf{Week 15} \\
    For the \textcolor{ta3scarletred}{H-Mode} scenario development we start from the best shot found in
    2016 (\# 33478) and increase the heating power
    \begin{enumerate}
      \item \textcolor{ta3scarletred}{Start from shot \# 33478 but
          with increased heating power (6MW). Adjust fueling rate from
        reference by increasing by a factor of 30 \%. 1 Plunge of
        probe head at the end of the discharge still in a safe
        position and IR monitoring}
    \item \textcolor{ta3scarletred}{Repeat \# 1 eventually adjusting
        the fueling rate. Start the N seeding in feedforward starting
        fro the level found in reference}
    \item \textcolor{ta3scarletred}{Trade off between \#1 and \#2
        Fueling/Seeding. Additional plunge of probe at the end of the
        discharge}
      \item \textcolor{ta3skyblue}{This scenario would allow the
          exploitation of particle accelleration (McClements) physics as piggy-back}
    \end{enumerate}
  }
  \only<4>{
  \textbf{Week 17: This strongly depends on the achievement of H-Mode
    scenario obtained in Week 15 } \\
    \begin{enumerate}
    \item \textcolor{ta3scarletred}{Repeat best H-Mode shot found in Week 15 1st Radial position of probe}
    \item \textcolor{ta3scarletred}{Repeat \#1, different probe position}
    \item \textcolor{ta3scarletred}{Repeat \#1, different probe
        position}
    \item \textcolor{ta3scarletred}{Repeat best H-Mode shot found in
        week 15 and reduced the cryopumps}
    \item \textcolor{ta3scarletred}{Repeat best H-Mode shot found in
        week 15 and puff from midplane}
    \item \textcolor{ta3scarletred}{Contingency}
    \item \textcolor{ta3scarletred}{Contingency}
    \end{enumerate}
  }
  \only<5>{
    Among the contingency we propose the following 4 possibilities to
    be discussed
    \begin{enumerate}
    \item \textcolor{ta3chameleon}{Reversing B$_t$ direction and repeat one identical shot (e.g \# 30269) to investigate
   the role of SOL flows in SOL shoulder formation and filamentary
   transport}
    \item \textcolor{ta3chameleon}{DN discharge with similar density ramps as in reference. Possibly the two X-point should
   sit on the same flux surface}
   \item \textcolor{ta3scarletred}{Attempt a scenario similar to Topic-06 which will be performed later in time. See for example
   shot \# 29816 (Presented by T. Eich in the GPM) which is at even
   higher power (8 MW) or \# 25740
   which is actually in DN (see proposal from J. Vicente). If we choose for this we could actually compare with the priority 3 of L-Mode
   contingency}
 \item \textcolor{ta3scarletred}{Reverse B$_{t}$ operation. In this case the L-H treshold is different and we might end
     by careful adjusting the power into I-Mode scenario}
   \item \textcolor{ta3skyblue}{Working at 2T in H-Mode would require
       additional development}
   \item \textcolor{ta3skyblue}{Given that the upper divertor is less
       diagnosed then the lower one we prefer eventually to operate
       with reverse B$_{\phi}$}
   
\end{enumerate}
  }
\end{frame}
\begin{frame}{Required diagnostic and analysis}
  \begin{itemize}
    \item[$\boxtimes$] Midplane Manipulator
    \item[$\square$] Li-Beam. \emph{Are fluctuations and profiles
        available simultaneously}
    \item[$\square$] RFA \#2
    \item[$\square$] Divertor probes
    \item[$\boxtimes$] Neutral profiles
    \item[$\square$] Infrared for probe head monitoring. \emph{Are Target
      infrared measurements available/useful?}
    \item[$\square$] GPI
    \item[$\boxtimes$] Reflectometer. \emph{The operation at 2T can be
      obtained during the q$_{95}$ scan}
    \item[$\boxtimes$] Fast probes on the limiter
    \item[$\square$] Bolometer/AXUV in the divertor region
  \end{itemize}
\end{frame}

\begin{frame}{Work to be accomplished before the experiment}
  \begin{enumerate}
    \item Check the shape modification during the current/q$_{95}$
      scan
    \item Probe conditioning?
    \item Check the status of the diagnostics including GPI (issue
      regarding the puffing)
    \item Methodology ($\Lambda_{div}$ computation and profile)
      $\lambda_n$ filaments properties etc.  
    \item Code preparation for analysis and
      visualization. \emph{GITHUB repository?}
    \item Optimization of the effort: please provide us a more
      detailed plan of your analysis in order to limit superposition
      or work duplication
  \end{enumerate}
\end{frame}
\end{document}

