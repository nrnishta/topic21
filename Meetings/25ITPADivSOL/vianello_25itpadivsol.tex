\documentclass[10pt, compress]{beamer}
\usetheme[conference=25th ITPA-DivSOL,venue=Chengdu, date=29/01/2018, titleprogressbar, logo=RFX-logo]{Eurof}
\usepackage{listings,amsmath,multimedia, amssymb}
\usepackage{tangocolors}
\usepackage{rfxcolor}
\definecolor{colorA}{HTML}{324D5C}
\definecolor{colorB}{HTML}{E37B40}
\definecolor{colorC}{HTML}{60A65F}
% multiple figure location
%\graphicspath{{/Users/vianello/Documents/Fisica/Conferences/IAEA/iaea2018/pdfbox/}{../Experiments/}}
% for drawing
\usepackage{pgf}
\usepackage{tikz}
\usetikzlibrary{arrows,shapes,backgrounds}
\usepackage{onimage}
\usepackage[export]{adjustbox}
% for font
\usepackage[absolute,overlay]{textpos}
  \setlength{\TPHorizModule}{1mm}
  \setlength{\TPVertModule}{1mm}

\usepackage[style=nature,citestyle=authoryear-comp,defernumbers=true,maxnames=2,firstinits=true,
uniquename=init,backend=bibtex8,arxiv=abs,mcite]{biblatex}
\bibliography{biblio}
\renewcommand*{\bibfont}{\footnotesize}
\renewcommand*{\citesetup}{\footnotesize}
\usepackage[export]{adjustbox}
\makeatother
\mode<presentation>
\makeatletter
% add a macro that saves its argument
\newcommand{\footlineextra}[1]{\gdef\insertfootlineextra{#1}}
\newbox\footlineextrabox
% for reducing font on a single slide
\newcommand\Fontvi{\fontsize{8}{7.2}\selectfont}
\title{SOL Filamentary transport: update from joint AUG-TCV MST1 experiment}
\date{29 January 2018}
\author[N.Vianello]{presented by N. Vianello on behalf of  MST1-Topic 21
  scientific team}
\begin{document}
\tikzstyle{every picture}+=[remember picture]
\maketitle
\begin{frame}{Scientific team}
  \centering{
  Volker Naulin, Matteo Agostini, Diogo Aguiam, Scott Allan, Matthias Bernert, Daniel Carralero Ortiz, 
Stefan Costea, Istvan Cziegler, Hugo De Oliveira, Joaquin Galdon-Quiroga, Gustavo Grenfell, Antti Hakola, 
Codrina Ionita-Schrittwieser, Heinz Isliker, Alexander Karpushov,
Jernej Kovacic, Benoît Labit, Bruce Lipschultz, 
Roberto Maurizio, Ken McClements, Fulvio Militello, Jeppe Miki Busk Olsen, Jens Juul Rasmussen, Timo Ravensbergen, Bernd Sebastian Schneider, Roman Schrittwieser, Jakub Seidl, Monica Spolaore, Christian Theiler, Cedric Kar-Wai Tsui, Kevin Verhaegh, Jose Vicente, 
Nickolas Walkden, Zhang Wei, Elisabeth Wolfrum, W. Vijvers}
j\end{frame}

\begin{frame}{Motivation and deliverables}
  \Fontvi
  \vspace{-1cm}
  \begin{itemize}
    \item \alert{Relation between downstream divertor conditions and
        up-stream SOL profiles is not well understood. Influence of
        SOL blob structures on shoulder formation and divertor
        conditions is key element towards predictive
        capabilities. Joint effort within the EUROfusion framework to
        address this issue on all the MST1 devices (AUG, TCV and MAST-U)}
  \end{itemize}
  \onslide<2->{A series of deliverables are foreseen for 2017-2018 program
\begin{enumerate}
\item \only<3>{\color{ta3chameleon}} Cross-machine L-Mode
  shoulder dependence on current both at constant B$_t$ and at
  constant $q_{95}$. Rationale: disentangle the effect of current and
  parallel connection length
\item \only<3>{\color{taorange}}Establish robust scenario for density
  shoulder profile in H-Mode and establish dependence on
  fuelling/neutral profiles/divertor condition
\item \only<3>{\color{ta3chameleon}}Fluctuations mesurement on AUG to study
  filamentary transport under high-power H-Mode conditions \only<3>{\color{consorziored}}and under
  different plasma configurations (SN, DN)
\item \only<3>{\color{taorange}}Study the role of ELM regimes,  neutral
  compression and particle density in filamentary transport and
  related shoulder formation
\item \only<3>{\color{taorange}} Identify the contribution of
  collisionality and seeding on filamentary transport and related
  shoulder formation
\item \only<3>{\color{consorziored}}Determine the effect of filaments and
  shoulder formation on target heat loads in different H-mode plasmas
\end{enumerate}}
\onslide<3>{
I will report only on few of the deliverables since part of them will
be studied in forthcoming campaigns. \textcolor{red}{Remember this is
  still a work in progress}
}
\end{frame}

\begin{frame}{Current scan at constant B$_t$ in L-Mode plasma}
  \begin{columns}
    \begin{column}{0.6\textwidth}
      \includegraphics<1>[width=\textwidth]{/Users/vianello/Documents/Fisica/Conferences/IAEA/iaea2018/pdfbox/EquilibriaIpScanConstantBt}
      \includegraphics<2>[width=\textwidth]{../../Experiments/AUG/analysis/pdfbox/GeneralIpScanConstantBt}
      \includegraphics<3>[width=.9\textwidth]{../../Experiments/TCV/analysis/pdfbox/CurrentScanConstantBt}
      \includegraphics<4>[width=\textwidth]{../../Experiments/AUG/analysis/pdfbox/NeutralsVsNe5IpConstantBt}
      \includegraphics<5>[width=\textwidth]{../../Experiments/AUG/analysis/pdfbox/NeutralsVsGreenwaldConstantBt}
      \includegraphics<6>[width=\textwidth]{/Users/vianello/Documents/Fisica/Conferences/IAEA/iaea2018/pdfbox/UpstreamTargetProfilesConstantBt}
      \includegraphics<7>[width=\textwidth]{/Users/vianello/Documents/Fisica/Conferences/IAEA/iaea2018/pdfbox/ExampleShoulderAmplitude}
      \includegraphics<8>[width=\textwidth]{../../Experiments/AUG/analysis/pdfbox/AmplitudeVsNe5IpConstantBt}
      \includegraphics<9>[width=\textwidth]{../../Experiments/AUG/analysis/pdfbox/AmplitudeVsGreenwaldIpConstantBt}      
      \includegraphics<10>[width=\textwidth]{../../Experiments/AUG/analysis/pdfbox/AmplitudeVsLambdaIpConstantBt}
      \includegraphics<11>[width=\textwidth]{/Users/vianello/Documents/Fisica/Conferences/IAEA/iaea2018/pdfbox/EfoldBlobAllColor}         
    \end{column}
    \begin{column}{0.4\textwidth}
      \begin{itemize}
        \item<1|only@1> Shape matched in within the single scan done for each of
          the machine
        \item<1|only@1> The scan implies a modification of the
          L$_{\parallel}$. AUG exhibit a parallel connection length
          which is 5 times smaller then TCV
        \item<2|only@2> AUG: Fueling reduced only at lower I$_p$ to
          avoid earlier disruption. Similar neutral pressure in the
          subdivertor region reached. NBI additional power added to
          keep power in the SOL approximately constant
        \item<3|only@3> TCV: Ohmic heating only. Similar neutral compression reached and
          D$_{\alpha}$ radiation from the floor. Ion flux rollover
          reached in all the three current,  although marginally at
          330 kA
        \item<4|only@4> Divertor neutral density estimated starting
          from D$_{\alpha}$ calibrated camera and using electron
          density and temperature from LP data. Neutral density
          increases earlier in edge density at lower current
        \item<5|only@5> Neutrals behavior reconciled whenever
          comparison considered as a function of Greenwald fraction  
        \item<6|only@6> For both AUG and TCV flattening of normalized
          upstream profile reached \alert{earlier in density at lower
            current.} For both the machine the increase of $\lambda_n$
          reached for larger values of $\Lambda_{div}$
        \item<7|only@7> Quantifying profile evolution using the
          \alert{shoulder amplitude metric} introduce by Wynn and
          Lipschultz for JET. \alert{Amplitude is the difference
            between normalized upstream density profiles}
        \only<8-9>{\item Amplitude evolve faster in density at lower
          current in the far SOL \onslide<9>{\alert{but once evolution \textit{vs} greenwald
            fraction is considered the evolution is equivalent between
            different current}}}
        \item<10|only@10> Amplitude evolution still reconciled in AUG if
          considered as a function of local evolution of $\Lambda_{div}$
        \item<11|only@11> For both AUG and TCV $\lambda_n$ increases
          with blob size without significant difference within the
          current explored
      \end{itemize}
    \end{column}
  \end{columns}
\end{frame}  

\begin{frame}{Current scan at constant q$_{95}$}
  \begin{columns}
    \begin{column}{0.6\textwidth}
      \includegraphics<1>[width=\textwidth]{/Users/vianello/Documents/Fisica/Conferences/IAEA/iaea2018/pdfbox/EquilibriaIpScanConstantQ95}
      \includegraphics<2>[width=\textwidth]{../../Experiments/AUG/analysis/pdfbox/GeneralIpScanConstantq95}
      \includegraphics<3>[width=.9\textwidth]{../../Experiments/TCV/analysis/pdfbox/CurrentScanConstantQ95}
      \includegraphics<4>[width=\textwidth]{../../Experiments/AUG/analysis/pdfbox/NeutralsVsGreenwaldConstantQ95}
      \includegraphics<5>[width=\textwidth]{/Users/vianello/Documents/Fisica/Conferences/IAEA/iaea2018/pdfbox/UpstreamTargetProfilesConstantQ95}
      \includegraphics<6>[width=\textwidth]{../../Experiments/TCV/analysis/pdfbox/CompareTargetProfilesConstantQ95}
      \includegraphics<7>[width=\textwidth]{../../Experiments/AUG/analysis/pdfbox/AmplitudeVsLambdaIpConstantQ95}
      \includegraphics<8>[width=\textwidth]{/Users/vianello/Documents/Fisica/Conferences/IAEA/iaea2018/pdfbox/EfoldVBlobAllColorConstantQ95}         
      \includegraphics<9>[width=\textwidth]{/Users/vianello/Documents/Fisica/Conferences/IAEA/iaea2018/pdfbox/BlobLambdaAllColorConstantQ95}         
    \end{column}
    \begin{column}{0.4\textwidth}
      \begin{itemize}
        \item<1|only@1> Shape matched in within the single scan even
          though this required for TCV operation with very low
          toroidal field (0.8T)
        \item<1|only@1> The parallel connection length remains almost unchanged
        \item<2|only@2> AUG: As for the case of constant B$_t$ we have
          pretty reproducible behavior matching basically the plasma
          condition in within the current scan
        \item<3|only@3> TCV: Even at such an high density at lower
          current (and lower B$_t$) no sign of target ion flux
          rollover/detachment
        \item<4|only@4> AUG: Divertor neutral density exhibits 
          the same behavior if considered as a function of greenwald fraction
        \item<5|only@5> For AUG upstream and target profiles still
          exhibit flattening earlier in density at lower current but
          \alert{always at large values of 
          $\Lambda_{div}$}. For TCV no sign of upstream profile flattening \alert{even
            at very large values of $\Lambda_{div}$}
        \item<6|only@6> This is due to the fact we did not reach
          divertor detachment which \alert{seems mandatory for
            upstream profile modification}  
        \item<7|only@7> AUG: Amplitude evolution as a function of $\Lambda_{div}$ 
          still reconcile the explored current scan
        \item<8|only@8> AUG exhibit consistently an increase of
          $\lambda_n$ with blob-size whereas for TCV the profile
          remains flat consistently with a small variation of
          $\delta_b$
        \item<9|only@9> And for TCV this is true even at high value of
          $\Lambda_{div}$. \alert{$\Lambda_{div}$ is not sufficient to
          guarantee flat profiles on TCV.}
      \end{itemize}
    \end{column}
  \end{columns}
\end{frame}
%% H-Mode same fueling
\begin{frame}{H-Mode analysis on AUG}
    \Fontvi
  \vspace{-1cm}
  \begin{columns}
  \begin{column}{0.55\textwidth}
    \centering{\includegraphics<1>[height=0.8\textheight]{../../Experiments/AUG/analysis/pdfbox/GeneralComparisonShot34276_34278_34281.pdf}}
    \centering{\includegraphics<2>[width=\textwidth]{../../Experiments/AUG/analysis/pdfbox/PuffingIpolsola34276_34278_34281}}
  \centering{\includegraphics<3>[height=0.75\textheight]{../../Experiments/AUG/analysis/pdfbox/UpstreamDivertorProfiles34276_34278_34281}}
  \centering{\includegraphics<4>[width=\textwidth]{/Users/vianello/Documents/Fisica/Conferences/IAEA/iaea2018/pdfbox/EfoldSizeShots34276_34278}}
  \centering{\includegraphics<5>[width=\textwidth]{/Users/vianello/Documents/Fisica/Conferences/IAEA/iaea2018/pdfbox/EfoldSizeShots34276_34278_34281}}
  \end{column}
  \begin{column}{0.45\textwidth}
    \begin{itemize}
    \item<1|only@1> We perform a series of shots in H-Mode with 6.5
      total heating power where we changed the fueling and the
      efficiency of cryopumps. Specifically we have
      \begin{itemize}
        \item<1|only@1>\textcolor{colorA}{\# 34276 without the
            cryompumps}
        \item<1|only@1> \textcolor{colorB}{\# 34278 with the same
            fueling as \# 34276 but with the cryompump}
        \item<1|only@1> \textcolor{colorC}{\#34281 where we increase
            fueling and seeding trying to mimic the same subdivertor
            pressure as \# 34276}
      \end{itemize}
    \item<1|only@1> Keeping the same fueling with the cryopump clearly
      reduce the pressure in the the sub-divertor area, we don't reach
      clear detachment and the edge density is constant even during
      the fueling ramp. Degraded H-mode reached later without the cryopump
      
    \item<2|only@2> Different behavior of ELM during the fueling
      ramp. ELM size and frequency changes strongly without the
      cryopump or during extreme fueling case
    \item<3|only@3> The profiles for shot \# 34278 with the cryopum
      and lower fueling remains more steep in all the three timing
      wind and the plasma is still attached. Interestingly for shot \#
      34281 with the cryopumps and higher fueling the detachment is
      more pronounced
    \item<4|only@4> Without the cryopumps, we reached flatter
         profiles with comparable inter-ELM resolved blob-size. This
         indicates strong neutral pressure effects in determining
         upstream profiles
    \item<5|only@5> Increasing the fueling and correspondingly the
      divertor neutral pressure move towards a situation similar to
      \# 34276 without the cryopump
      \end{itemize}
  \end{column}
\end{columns}
\end{frame}

\begin{frame}{Conclusion}
  \Fontvi
  \begin{itemize}
    \item Current scan at constant B$_t$ and at constant q$_{95}$
      performed during density ramps L-Mode experiments both at AUG
      and TCV
    \item In both the case shoulder appear earlier in density at lower
      current but AUG shows reconciliation of behavior if considered
      as a function of greenwald fraction and $\Lambda_{div}$
    \item Both the experiments exhibit at constant B$_t$ flattening of the profile as
      blob size is increasing, independently from the current. The
      same behavior is observed during current scan at constant
      q$_{95}$ \alert{only on AUG}
    \item On TCV during the current scan at constant q$_{95}$
      detachment not reached and this \alert{prevent upstream profile
        flattening}
    \item H-Mode experiments performed on AUG where fueling and
      pumping have been varied. Proved inter-ELM profile flattening
      also in H-Mode with a more efficient puffing without the
      cryopump. \alert{Hints on the role of neutrals also in H-Mode}
  \end{itemize}
\end{frame}

\begin{frame}{D$_{\alpha}$ tomography: work in progress}
\Fontvi
  \begin{columns}
    \begin{column}{0.5\textwidth}
      \includegraphics<1>[width=\textwidth]{../../Experiments/AUG/analysis/pdfbox/Dalpha/inv_alfa_34104_2000}
      \includegraphics<2>[width=\textwidth]{../../Experiments/AUG/analysis/pdfbox/Dalpha/inv_alfa_34104_2700}
      \includegraphics<3>[width=\textwidth]{../../Experiments/AUG/analysis/pdfbox/Dalpha/inv_alfa_34104_3100}
    \end{column}
    \begin{column}{0.5\textwidth}
      \begin{itemize}
        \item \alert{Preliminary} results from inversion tomography
          from calibrated D$_{\alpha}$ camera under the assumption of
          toroidally symmetric emission limited to the region outside
          the LCFS
        \item \emph{Simultaneous Algebraic Reconstruction Technique
            (SART) (A.H. Anderson and A.C.Kak,  Ultrasonic Imaging
            \textbf{6},  81 (1984))}
        \item<2-> As time evolves during the L-Mode ramp we observe
          D$_{\alpha}$ radiation moving from HFS towards the LFS
          common flux region
        \item<3-> \alert{This is consistent with JET observation in
            Horizontal target regime whenever shoulder is observed}
      \end{itemize}
    \end{column}
  \end{columns}
\end{frame}


\end{document}

