\documentclass[10pt, compress]{beamer}
\usetheme[conference=TCV-Topic21 KoM,venue=Lausanne, date=15/05/2017, titleprogressbar, logo=RFX-logo]{Eurof}
\usepackage{listings,amsmath,multimedia, amssymb}
\usepackage{../beamerclass/tangocolors}
\usepackage{../beamerclass/rfxcolor}
% for drawing
\usepackage{pgf}
\usepackage{tikz}
\usetikzlibrary{arrows,shapes,backgrounds}
\usepackage{../beamerclass/onimage}
\usepackage[export]{adjustbox}
\usepackage{bm}
% for font
\usepackage[absolute,overlay]{textpos}
  \setlength{\TPHorizModule}{1mm}
  \setlength{\TPVertModule}{1mm}

\usepackage[style=nature,citestyle=authoryear-comp,defernumbers=true,maxnames=2,firstinits=true,
uniquename=init,backend=bibtex8,arxiv=abs,mcite]{biblatex}
\bibliography{biblio}
\renewcommand*{\bibfont}{\footnotesize}
\renewcommand*{\citesetup}{\footnotesize}
\usepackage[export]{adjustbox}
\makeatother
\mode<presentation>
\makeatletter
% add a macro that saves its argument
\newcommand{\footlineextra}[1]{\gdef\insertfootlineextra{#1}}
\newbox\footlineextrabox
% for reducing font on a single slide
\newcommand\Fontvi{\fontsize{8}{7.2}\selectfont}
\title{Topic 21 experiment Week 24}
\date{26 June 2017}
\author[N. Vianello,  V. Naulin]{N . Vianello and V. Naulin for the
  Topic 21 SC team}
\begin{document}
\tikzstyle{every picture}+=[remember picture]
\maketitle
\begin{frame}{Scientific Team}
  \begin{description}
  \item[EPFL:] H. De Oliveira, R. Maurizio, B. Labit,
    C. Tsui, K. Verhaegh, H. Reimerdes, C. Theiler
  \item[DTU:] J.J. Rasmussen and V. Naulin
  \item[RFX:] N. Vianello, M. Spolaore, M. Agostini
  \item[OEAW:] B. Schneider, S. Costea, R. Schrittwieser
  \item[CCFE:] F. Militello
  \item[JSI:] J. Kovacic
  \end{description}
\end{frame}
\begin{frame}{TCV experiments: boundary condition}
\vspace{-1cm}
\Fontvi
\begin{itemize}
\item 2017 \textbf{objectives} listed after the General Planning Meeting
  \begin{enumerate}
  \item Provide cross-machine \alert{L-Mode} shoulder dependence on
    current both at constant Bt and at constant q$_{95}$
  \item Establish robust scenario for density shoulder profile in
        H-mode and establish dependence on fuelling/neutral
        profiles/divertor condition
  \item Study the role of ELM regimes, neutral compression, and
    particle density in filamentary transport and related shoulder
    formation.
  \item Identify the contribution of collisionality and
    seeding on filamentary transport and related shoulder
    formation.
  \item Determine the effect of filaments and shoulder
    formation on target heat loads in different Hmode plasmas.
  \end{enumerate}
\item We have a total number of \textbf{\# 23 Shots} originally split into two
    operational window. Calendar week 24 (12.06-16.06) and Calendar
    week 43 (23.10-27.10)
\end{itemize}
\end{frame}

\begin{frame}{Experimental plan}
  \begin{frame}{Proposed experimental plan}
\Fontvi
    \alert{For the first week of operation we originally planned L-mode shots
      only. Shots 1-3 I$_p$ scan at constant toroidal field. Shots
      1, 4 and 5 I$_p$ scan at constant q$_{95}$ to be compared with
      analogous experiments in AUG and MAST-U. Shots 6-7 Low
      collisionality scan. Shots 8-9 DN current scan: this will be
      compared directly with Mast-U which will run predominantly in DN
    configuration. Shot 10-11 Current scan in forward field to check
    the role of $\nabla\times B$ direction.}
\begin{enumerate}
\item Shape from 57088, I$_p$ = 245 kA,  Reverse B$_t$,
    density ramp from Line Average Density = 3.8e19 @ 0.5 s to 11e19 @ 1.6s,  Bt = 1.4T. Plunge @ 0.65, 1.52
\item  Repeat \# 1 with I$_p$=330 kA Bt=1.4T, same density ramp, same timing for plunges
\item  Repeat \# 1 with I$_p$=180 kA, Bt=1.4T, same density ramp, same timing for plunges
\item  Repeat \# 1 with q95=2.44 as \# 2, adjust Bt consequently (Bt = 1.02T)
\item  Repeat \# 3 with q95=2.44 as \# 2, adjust Bt consequently (Bt=0.8T)
\item  Shape and current from \# 1. Stop puffing once the divertor is
  formed to get low collisionality case. ECRH ramp from 0.9s (150
  kW--500 kW)
\item  Repeat \# 6 with intermediate density value between \# 6 and
  \#1 density at 0.65s. 
\item  Repeat density ramp of Shot \# 2 in DN configuration 
\item  Repeat density ramp of Shot \# 3 in DN configuration 
\item Repeat \# 1 in forward field
\item Repeat \# 3 in forward field
\end{enumerate}
\end{frame}

\begin{frame}{Current scan at constant Magnetic field}
  \begin{columns}[c]
    \begin{column}{0.65\textwidth}
      \includegraphics[width=\textwidth]{../../Experiments/TCV/analysis/pdfbox/}
    \end{column}
  \end{columns}
  

\begin{frame}{Results obtained during MST1 Campaigns 2015-2016}
  \begin{columns}
    \begin{column}{0.6\textwidth}
      \includegraphics<1>[width=\textwidth]{pdfbox/KoM150517/Fig0}
      \includegraphics<2>[width=\textwidth]{pdfbox/KoM150517/Fig1}
      \includegraphics<3>[width=\textwidth]{pdfbox/KoM150517/Fig2}
      \includegraphics<4>[width=\textwidth]{pdfbox/KoM150517/Fig4}
      \includegraphics<5>[width=\textwidth]{pdfbox/KoM150517/Fig5}
      \includegraphics<6>[width=\textwidth]{pdfbox/KoM150517/Fig6}
      \includegraphics<7>[height=.8\textheight]{pdfbox/KoM150517/Fig7}
      \includegraphics<8>[height=.8\textheight]{pdfbox/KoM150517/Fig8}
    \end{column}
    \begin{column}{0.4\textwidth}
      \begin{itemize}
        \item<1|only@1> Performed a series of L-Mode shots with
          density ramp and different poloidal flux expansion to check
          L$_{\parallel}$ effect
        \item<2|only@2> Upstream profiles at same densities but
          different $\Lambda_{div}$ (well above 1 all along the
          profiles) are similar. Weak effect of parallel connection
          length modification
        \item<3|only@3> Density decay length $\lambda_n$ modified
          strongly with increasing density in the near SOL region
        \item<4|only@4> Blob size increases with density independently
          from the Flux expansion
        \item<5|only@5> In the Far SOL the decay length scales with
          divertor collisionality but examples with high $\Lambda_n$
          and steep profile clearly exhists
        \item<6|only@6> Steep profile clearly seen in the
          Sheath-connected regime. Unclear if Resistive balloning and
          resistive X-point blobs behaves differently
        \item<7|only@7> Change in the Flux expansion does not change
          detachment density threshold, whereas divertor leg length
          has a clear impact (\textit{Theiler, NF17 and Reimerdes IAEA
            2016})
        \item<8|only@8> Also any differences observed between single
          and double null. Major modification only induced by density increase  
          
      \end{itemize}
    \end{column}
  \end{columns}
\end{frame}


\begin{frame}{Proposed experimental plan: L-mode}
  \Fontvi
  The first shots are needed in order to have comparable scenario
  between all the machines
  \begin{itemize}
    \item Current scan at constant q$_{95}$
      \begin{enumerate}
      \item Reference shot is \# 54867, I$_p$=240 kA, q$_{95}$=3.5,
        B$_t$ = 1.41T with the same density ramp (\alert{Do we need a density ramp? can we
          go directly at high density as in \# 53516? and repeat the
          shot at low density?})
      \item Repeat Shot \# 1 with I$_p$=180 kA (20\% decrease similarly
        to AUG) with the same q$_{95}$ and adjusting the toroidal field
      \item Repeat Shot \# 1 with I$_p$=300 kA (20\% increase similarly
        to AUG) with the same q$_{95}$ and adjusting the toroidal
        field. \alert{In case this causes a transition into Ohmic H-Mode we
        might think to perform the shot in reverse B$_t$. This
        eventually needs additional shots at lower current as
        well. \emph{Why not operate all the shots in Reverse B$_t$?}}
      \end{enumerate}
    \item Current scan at constant B$_t$
      \begin{enumerate}
        \setcounter{enumi}{3}
      \item I$_p$=180 kA,  B$_t$ = 1.41 as in shot \# 1 
      \item I$_p$=300 kA, B$_t$=1.41 as in shot \# 1
      \end{enumerate}
    \end{itemize}      
  \end{frame}

  \begin{frame}{Proposed experimental plan: L-Mode}
    \begin{itemize}
      \item Divertor leg length/L$_{parallel}$ scan
      \begin{enumerate}
        \setcounter{enumi}{5}
      \item Repeat \# 1 at $Z=+23$.
      \item Repeat \# 1 at $Z=-10$  
      \end{enumerate}
      \item This can have issues in terms of probe operation because
        of shadowing effect. 
      \item Double Null. Repeat the current scan at constant q$_{95}$ in
        DN configuration in order to provide suitable comparable
        scenario with foreseen MAST-U operation
        \begin{enumerate}
          \setcounter{enumi}{7}
          \item Repeat the density ramp of \# 1 in DN configuration
            I$_p$ = 240kA
          \item Repeat density ramp of shot \# 2 in DN configuration
            I$_p$ = 180 kA
          \item Repeat density ramp of shot \# 2 in DN configuration
            I$_p$ = 300 kA
          \end{enumerate}
        \end{itemize}
      \end{frame}

      \begin{frame}{Proposed experimental plan: H-Mode}
        The NBI heated plasma is still a partially uncovered scenario
        for TCV. We have a good reference shot \# 53352 with a good
        type-I Elmy regime. We propose to use the 3 shots for proper
        scenario development
        \begin{enumerate}
          \setcounter{enumi}{10}
          \item Repeat \# 53352 same setting. 1MW NBI power 0.4
            s starting at 0.8s. At 0.82s start a density ramp keeping
            the same rate as in \# 1. Power ramp down from 1.2 second
            in order to check ELMy regime. \alert{We need a good
              monitoring of divertor condition and modification}
          \item The second shot will depend on \# 11. We will need to
            adjust the fueling rate accordingly. Include N seeding
            \alert{check for appropriate reference from past
              experiment}
          \item Repeat \# 12 with best trade off between seeding and fueling
          \end{enumerate}
          The hypothesis to reach ECRH high density H-Mode is limited, to my
          knowledge, to operation in vertically shifted plasma and
          X3. Can be explored in W23
        \end{frame}

        \begin{frame}{Diagnostic and open issues}
          %\Fontvi
          \begin{itemize}
          \item Fast Camera
          \item DBS radial localization
          \item Neutrals from gauges. D$_{\alpha}$ calibrated camera
            to be used together with \texttt{KN1D} code
          \item Fast reciprocating probe. Can it work with radially
            spaced I$_s$ measurements?
          \item Do we miss something in L-Mode which is worth to be
            done?
          \item Do we have experience of nitrogen seeded NBI heated
            discharge?
          \item Do we have other references for high density H-Mode operation?
          \end{itemize}
        \end{frame}
\end{document}

