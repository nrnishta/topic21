\documentclass[10pt, compress]{beamer}
\usetheme[conference=MST1-EPM,venue=Garching, date=15/11/2016, titleprogressbar, logo=RFX-logo]{Eurof}
\usepackage{listings,amsmath,multimedia, amssymb}
\usepackage{../beamerclass/tangocolors}
\usepackage{../beamerclass/rfxcolor}
% for drawing
\usepackage{pgf}
\usepackage{tikz}
\usetikzlibrary{arrows,shapes,backgrounds}
\usepackage{../beamerclass/onimage}
\usepackage[export]{adjustbox}
\usepackage{bm}
% for font
\usepackage[absolute,overlay]{textpos}
  \setlength{\TPHorizModule}{1mm}
  \setlength{\TPVertModule}{1mm}

\usepackage[style=nature,citestyle=authoryear-comp,defernumbers=true,maxnames=2,firstinits=true,
uniquename=init,backend=bibtex8,arxiv=abs,mcite]{biblatex}
\bibliography{biblio}
\renewcommand*{\bibfont}{\footnotesize}
\renewcommand*{\citesetup}{\footnotesize}
\usepackage[export]{adjustbox}
\makeatother
\mode<presentation>
\makeatletter
% add a macro that saves its argument
\newcommand{\footlineextra}[1]{\gdef\insertfootlineextra{#1}}
\newbox\footlineextrabox
% for reducing font on a single slide
\newcommand\Fontvi{\fontsize{8}{7.2}\selectfont}
\title{Filamentary transport in high-power H-mode conditions and in
  no/small-ELM regimes to predict heat and particle loads on PFCs for
  future devices}
\date{15 November 2016}
\author[Topic 21. J. Madsen and N.Vianello]{J. Madsen and N. Vianello}
\begin{document}
\tikzstyle{every picture}+=[remember picture]
\maketitle

\begin{frame}{Main focus of topic as listed in MST1 call}
\vspace{-1cm}
Objectives:
\begin{enumerate}
\item Use the new HHF probe on AUG to study filamentary transport under
high-power H-mode conditions and under different plasma configurations (SN,
DN).
\item Study the role of ELM regimes, neutral compression, and particle density in
filamentary transport and related shoulder formation.
\item Identify the contribution of collisionality and seeding on filamentary transport
and related shoulder formation.
\item Extend the studies to quiescent H-modes as well as to other small-ELM regimes.
\item Determine the effect of filaments and shoulder formation on target heat loads
in different H-mode plasmas.
\item Investigate the effect of plasma shape and configuration on ELM-induced heat
loads.
\end{enumerate}
\underline{Motivation}: Particle and energy transport in the SOL is crucial for the lifetime of plasma facing components in ITER and DEMO
\end{frame}

\begin{frame}{Outline}
	\begin{itemize}
		\item Background 
		\item Plans (we will not make a detailed list of the
                  proposal, we already tried to combine and summarize them)
		\item Discussion
	\end{itemize}	
\end{frame}

\section{Background}
\begin{frame}{Present status}
  \setbeamercovered{transparent}
  \begin{enumerate}[<+(1) | invisible@-+>]
    \item Experimental and theoretical investigations suggest that SOL density profiles are
      determined by the balance between radial filamentary transport and parallel dynamics
    \item The balance changes at high density and leads to broadened SOL density profiles $\rightarrow$ \emph{shoulder formation}
    \item This is observed in L-mode plasmas in a variety of devices including AUG \parencite{Carralero:2014gs},
      TCV \parencite{Garcia:2007p2615} and MAST \parencite{Militello:2016hk}
    \item Observations exhibit various other dependencies e.g., divertor
      collisionality, plasma current, seeding. \alert{Results from
        different devices not fully reconciled and the underlying mechanism is not fully understood.}
    \item Studies of shoulder formation in H-Mode are so far limited 
	%\item Particle and energy transport in the SOL is crucial for lifetime of PFC in ITER and DEMO
%    \item First we will report a summary of the achievements and then
 %    personal comments and plans
    \end{enumerate}
\end{frame}
\section{L-Mode}
\begin{frame}{L-Mode studies: AUG/1}
  \begin{columns}
    \begin{column}{0.5\textwidth}
      \begin{itemize}
      \item<1-> AUG and JET \parencite{Carralero:2015gu} suggest that divertor collisionality 
        \begin{equation*}
        \Lambda_{div} = \frac{L_{\parallel}/c_s}{1/\nu_{ei}}\frac{\Omega_i}{\Omega_e}
      \end{equation*}
      controls filaments size
        \item<2-> \textcolor{red}{Tested by changing n$_e$
            and $T_e$ through fueling/seeding/heating}
        \item<3-> This determines a change of the velocity-size
          scaling from
            \textcolor{ta3chameleon}{sheath-limited}
        to \textcolor{blue}{inertial regime}.
       \onslide<4->{\alert{$\Lambda_{div}$ rules the density profile scale
           length}}
       \item<5> This finally determines the shoulder formation
        \end{itemize}
    \end{column}
    \begin{column}{0.5\textwidth}
      \only<1-2>{
      \begin{tikzonimage}[width=\textwidth]{../pdfbox/KoM15Nov/CarraleroPRL15a}
          \draw [thick, ta3chameleon, thick, rotate=1] (0.43,0.25) ellipse (0.23 and 0.1);
          \draw [thick, blue, thick, rotate=-4] (0.8,0.55) ellipse (0.1 and 0.33);
        \only<2>{
          \draw [ultra thick, red, thick, dashed] (0.5,0.63) circle (0.12);
        }
      \end{tikzonimage}}
    \only<3-4>{
      \begin{tikzonimage}[width=\textwidth]{../pdfbox/KoM15Nov/CarraleroPRL15b}
          \draw [ultra thick, ta3chameleon, rotate=1] (0.2,0.5) ellipse (0.05 and 0.4);
          \draw [thick, blue, rotate=15] (0.65,0.07) ellipse (0.4 and 0.1);
      \end{tikzonimage}
    }
    \includegraphics<5>[width=\textwidth]{../pdfbox/KoM15Nov/CarraleroPRL15c}
    \end{column}
  \end{columns}
\end{frame}


\begin{frame}{L-Mode studies: AUG/2}
  \begin{columns}
    \begin{column}{0.5\textwidth}
      \begin{itemize}
      \item<1-> Profile modified by an increase of blob-size and change of
        \alert{packing fraction: $f_{fil} = \nu_{fil}\tau_{AC}$}
        \onslide<2->{\textcolor{blue}{and filament relative density}
          {\footnotesize (Carralero 2016 in preparation)}}
      \item<3-> As a consequence the contribution of filaments to
        radial transport increases
      \item<4|only@4> \alert{Beware the change of frequency may be due to a
        modification of fluctuation velocity which is known to vary
        with densities ad normalized greenwald fraction \parencite{0029-5515-51-5-053020}}  
      \item<5> Parallel flow is strongly reduced whenever we increase
        the divertor collisionality
      \end{itemize}
    \end{column}
      \begin{column}{0.5\textwidth}
        \includegraphics<1>[width=\textwidth]{../pdfbox/KoM15Nov/CarraleroMST16a}
        \includegraphics<2>[width=\textwidth]{../pdfbox/KoM15Nov/CarraleroMST16b}
        \includegraphics<3>[width=\textwidth]{../pdfbox/KoM15Nov/CarraleroMST16c}
        \includegraphics<4>[width=\textwidth]{../pdfbox/KoM15Nov/AgostiniNF2011}
        \includegraphics<5>[width=\textwidth]{../pdfbox/KoM15Nov/CarraleroMST16d}
      \end{column}
    \end{columns}
  \end{frame}

  \begin{frame}{L-Mode studies:AUG/3}
  \begin{columns}
    \begin{column}{0.5\textwidth}
      \begin{itemize}
      \item<1-> Electron and ions behave differently
      \item<2|only@2> T$_{e, fil} \sim 1.2 $T$_{e, bk}$ roughly constant
        accross the SOL and slightly affected by the increase of
        divertor collisionality
      \item<3|only@3> Ions are strongly affected: for
        \textcolor{red}{$\Lambda_{div}<1$ T$_{i, fil} > $ T$_{i, bk}$
          and $\lambda_{T_i} \sim 30$
          mm}. \textcolor{blue}{$\Lambda_{div}>1$ T$_{i, fil} \sim $
          T$_{i, bk}\sim 25$ eV
          and $\lambda_{T_i} \sim 8$ mm}
       \item<4-> Ion energy spectrum from
         $\mathbf{E}\times\mathbf{B}$ analyzer shrinks towards lower
         energy for $\Lambda_{div} > 1$
       \item<5> EMC3-Eirene simulation suggests that such a reduction
         can't be accounted for thermalization process. An
         \alert{ionization front builds in front of the limiter shadow}   
       \end{itemize}
    \end{column}
      \begin{column}{0.5\textwidth}
        \includegraphics<2>[width=\textwidth]{../pdfbox/KoM15Nov/CarraleroMST16e}
        \includegraphics<3>[width=\textwidth]{../pdfbox/KoM15Nov/CarraleroMST16f}
        \includegraphics<4>[width=\textwidth]{../pdfbox/KoM15Nov/CarraleroMST16g}
        \includegraphics<5>[width=\textwidth]{../pdfbox/KoM15Nov/CarraleroMST16h}
      \end{column}
    \end{columns}
  \end{frame}

  \begin{frame}{L-Mode: TCV}
    \begin{columns}
    \begin{column}{0.5\textwidth}
      \begin{itemize}
      \item<1|only@1> Flexibility has allowed to test $\Lambda_{div}$
        dependence on L$_{\parallel}$ by varying flux expansion f$_x$:
        \begin{equation*}
          f_x = \frac{(B_p/B_t)_{MP}}{(B_p/B_t)_{SP}}
        \end{equation*}
        in ohmic density ramps \parencite{vianello:iaea2016}
      \item<2|only@2> Slight variation of density profiles at the target but
        due to direct dependence on L$_{\parallel}$ large increase of
        $\Lambda_{div}$. \alert{Upstream profiles only varies whenever
        we reach a certain amount of $\langle n_e \rangle$}
       \only<3-5>{\item Weak dependence of blob-size from $\Lambda_{div}$,
        \onslide<4-5>{\textcolor{red}{also on a statistical
            basis}}. \onslide<5>{Strong dependence on average density,
          independent of L$_{\parallel}$}}
       \item<6|only@6> $\lambda_n$ depends clearly on blob-size
         whereas the dependence on divertor condition is less
         obvious. \alert{$\Lambda_{div}$ necessary but not sufficient}
      \end{itemize}
    \end{column}
      \begin{column}{0.5\textwidth}
        \includegraphics<2>[width=\textwidth]{../pdfbox/KoM15Nov/VianelloIAEA16a}
        \includegraphics<3>[width=\textwidth]{../pdfbox/KoM15Nov/VianelloIAEA16b}
        \includegraphics<4>[width=\textwidth]{../pdfbox/KoM15Nov/VianelloIAEA16c}
        \includegraphics<5>[width=\textwidth]{../pdfbox/KoM15Nov/VianelloIAEA16d}
        \includegraphics<6>[width=\textwidth]{../pdfbox/KoM15Nov/VianelloIAEA16e}
      \end{column}
    \end{columns}
  \end{frame}

  \begin{frame}{L-Mode: JET}
    \begin{columns}
    \begin{column}{0.5\textwidth}
      \begin{itemize}
      \item<1|only@1> The shoulder formation strongly depends on
        divertor geometry, disappear with vertical target and strike
        point closest to cryogenics pumps \parencite{Wynn:EPS2016}
      \item<2|only@2> The midplane pressure from baratrons is
        equivalent between the different divertor. \alert{This would
          indicate that SOL neutral density at the outboard midplane does not play any role}
       \item<3|only@3> In the horizontal target configuration the
         results indicate that the shoulder forms right at the
         transition from sheath-limited to high-recycling where also
         $\Lambda_{div}$ strongly increase
       \item<4|only@4> Shoulder amplitude correlates with strike
         points position. \alert{Shoulder, ionization and
           $\Gamma_{ion, plate}$ larger when R$_{strike}$ smaller away
         from the pump}
       \item<5|only@5> In seeded discharges the transition observed at
         very high level of $\Lambda_{div} >> 1$
      \end{itemize}
    \end{column}
      \begin{column}{0.5\textwidth}
        \includegraphics<1>[width=\textwidth]{../pdfbox/KoM15Nov/LipschultzITPA16a}
        \includegraphics<2>[width=\textwidth]{../pdfbox/KoM15Nov/LipschultzITPA16b}
        \includegraphics<3>[width=\textwidth]{../pdfbox/KoM15Nov/LipschultzITPA16c}
        \includegraphics<4>[width=\textwidth]{../pdfbox/KoM15Nov/LipschultzITPA16d}
        \includegraphics<5>[width=\textwidth]{../pdfbox/KoM15Nov/LipschultzITPA16e}
      \end{column}
    \end{columns}
  \end{frame}

  \begin{frame}{L-Mode: MAST}
    \begin{columns}
    \begin{column}{0.5\textwidth}
      \begin{itemize}
      \item<1|only@1> Strong dependence on
        I$_p$ \parencite{Militello:2016hk}. Increasing I$_p$ at
        constant toroidal field shoulder disappears. Consistent with
        observation in other devices
      \item<2|only@2> Filaments binormal dimension increases with
        current \parencite{Kirk:2016jj} or equivalently decreases with L$_{\parallel}$
       \item<3|only@3> Filament radial velocity decreases with current
         as well as the radial dimension \parencite{Kirk:2016jj}
      \end{itemize}
    \end{column}
      \begin{column}{0.5\textwidth}
        \includegraphics<1>[width=\textwidth]{../pdfbox/KoM15Nov/MilitelloNF16a}
        \includegraphics<2>[width=\textwidth]{../pdfbox/KoM15Nov/KirkPPCF16a}
        \includegraphics<3>[width=\textwidth]{../pdfbox/KoM15Nov/KirkPPCF16c}
      \end{column}
    \end{columns}
  \end{frame}

  \section{H-Mode}
  \begin{frame}{AUG: H-Mode}
    \begin{columns}
    \begin{column}{0.4\textwidth}
      \begin{itemize}
      \item<1|only@1> SOL profiles in H-Mode so far investigated on
        AUG \parencite{Muller:2015jt,Sun:2015bu,carralero:psi2016}
      \item<2|only@2> Differently from L-Mode, complete detachment
        suggested to be mandatory for increasing of $\lambda_n$ \parencite{Sun:2015bu}
       \item<3|only@3> In weak H-Mode \parencite{carralero:psi2016}
         shoulder depends on a combination of $\Lambda_{div}$ and
         fueling rate. Complete detachment seems instead not necessary
         although very weak shoulder
      \end{itemize}
    \end{column}
      \begin{column}{0.6\textwidth}
        \includegraphics<1>[width=\textwidth]{../pdfbox/KoM15Nov/MuellerJNM15a}
        \includegraphics<2>[width=\textwidth]{../pdfbox/KoM15Nov/SunPPCF15a}
        \includegraphics<3>[width=\textwidth]{../pdfbox/KoM15Nov/CarraleroMST16i}
      \end{column}
    \end{columns}
  \end{frame}

  \section{Open issues}
  \begin{frame}{Open and unresolved issues}
  \setbeamercovered{transparent}
  \begin{enumerate}[<+(1) | invisible@-+>]
    \item Does $\Lambda_{div}$ unify the picture among the devices?
      \alert{No, does neutral pressure play a role?}
    \item If neutrals play a role is it at the divertor and/or at the midplane?
      \alert{Contradictory results if one include experiments with
        midplane puffing, JET and EIRENE simulations}
    \item Does L$_\parallel$ play any role? \alert{So far TCV suggests no
      dependence from f$_x$ scan but both TCV and MAST observe an
      I$_p$ dependence. MAST clearly state filament $\sigma_{\perp}$
      increases with L$_{\parallel}$ }
    \item Is cooling the divertor with fueling or seeding equivalent? \alert{Contradictory observations in AUG and JET}
    \item Do we observe same behavior in L and H-Mode? \alert{So
      far no as shown in H-Mode AUG. We need higher detachment
      condition and we need enough fueling}
      \item Is the density shoulder accompanied with an ion temperature shoulder?
%      \item What are the underlying mechanisms? 
    \end{enumerate}
  \end{frame}
  
\section{Topic 21 experiments }
\begin{frame}{$n_{\text{proposed shots}} \gg n_{\text{allocated shots}}$}
	\begin{itemize}
		\item 15 were proposals submitted to Topic 21
		\item Proposals include experiments on all three machines 
		\item There are overlaps between several of the proposals 
		\item Preliminary shot allocation. AUG: 14. MAST: 13. TCV: 23 
		\item However, total number of proposed shots:  \textbf{449}
		\item Several of the proposed experiments can be combined. \underline{But we must prioritize}
		\item Reaching all goals is not possible with the current number of allocated shots.
	\end{itemize}
\end{frame}

\begin{frame}{Focal points}
\begin{itemize}
	\item MST1 uniquely facilitates cross comparison between machines
	\item Several of the proposed experiments overlap and will be combined
	\item A cross machine experiment has the makings of settling open issues
	\item Therefore, we will allocate shots for cross-machine
          comparison but we must strive after similar machine configuration
	\item ....but also to other proposed experiments
	\item Cross machine L-mode experiments: 
	\begin{enumerate}
		\item Investigate the role of neutrals 
		\item $I_p$ and $q_{95}$ scans
	\end{enumerate}
	\item Cross machine H-mode experiments. 
\end{itemize}	
\end{frame}

\begin{frame}{L-mode experiments}
	\begin{itemize}
		\item The role of neutrals in the shoulder
                  formation is not understood
                  {\scriptsize(proponents: Carralero (\# 7), Militello
                    (\# 1, 2), Vianello (\# 13), Walkden (\# 9))}
		\begin{enumerate}
		\item We envisage to measure neutral gas profiles at the outboard midplane
                  using fast cameras with D$_{\alpha}$ filter, density
                  and temperature SOL profiles and appropriate code
                  (e.g. KN1D as in \parencite{Lipschultz:2005gg,Lipschultz:2016vk})
		\item Reciprocating probes and profiles available on
                  all machines
                \item Investigate role of fueling location (MAST)
		\end{enumerate}
		\item Disentangle the roles of $I_p$, $q_{95}$, and $L_{\|}$.
                  {\scriptsize(proponents: Carralero , Militello, Vianello, Tsui, Carralero (\# 7), Militello
                    (\# 1, 2), Vianello (\# 13), Walkden (\# 9), Tsui
                    (\# 15))}
		\begin{enumerate}
			\item Carry out parameter scans on all machines
			
		\end{enumerate}
	\end{itemize}
\end{frame}

\begin{frame}{H-mode experiments}
\begin{itemize}
	\item ITER and DEMO will operate in H-mode
	\item We must know what parameters control shoulder formation 
	\item Shoulder formation parameter regime is unclear on all machines
	\item Main priorities: 
	\begin{enumerate}
		\item Investigate if clear shoulder formation exists and what plasma parameters required? 
		\item Investigate the SOL (filamentary) transport properties.
                  Whenever main diagnostic is reciprocating probes $\rightarrow$ limits heating power
		\item Experiments must gradually increase power and density. 
	\end{enumerate}
	\item Fueling in H-mode is problematic due to transport
          barrier. Is there a way to overcome it? NBI, pellet fueling?
	\item H-mode density limit \parencite{Bernert:2015bq} must be dealt with
\end{itemize}

\end{frame}


\begin{frame}{Optimizing cross machine comparison}
\begin{itemize}
	\item Machines are fundamentally differently designed
	\item Strive for similar configurations of the machines:
	\begin{enumerate} 
		\item Single-Null
		\item $\bm{B} \times \nabla B$ towards active divertor 
		\item Strike-point and cryo pump location (AUG and MAST-U)
		\item Heating 
		\item Fueling location 
		\item Seeding (species, location, rate)
	\end{enumerate}
\end{itemize} 
%things we cannot influence but could be important : wall (carbon, tungsten), helicity, aspect ratio, divertor design, 


\end{frame}

\begin{frame}{Required diagnostics and competences}
In order to compare experiments the following diagnostics must be available: 
\begin{enumerate}
	\item Filaments analysis in the Outboard midplane: 
	\begin{itemize} 
		\item Through Reciprocating probe then $I_{sat}$  on minimum three pins poloidally and radially separated
		 (filaments speed and size). 
		\item Electron temperature (fast T$_e$ measurements
                  can be achieved with BPP in MAST-U {\footnotesize
                    (N. Walkden \#11)})
		\item If possible $M_{\|}$
	 \end{itemize}
	 \item Camera viewing OM for measuring neutrals 
	 \item Divertor measurements of $n$ and $T_e$ (probes) plus
           spectroscopy and bolometry 
	 \item Density profile measurements (Li-Bes, Reflectometry (\#
           6 Acquiam, \# 8 Vicente), Edge Thomson scattering) 
         \item Sami and Field for fast particles accelleration (piggy
           back on developed scenarion \# 4 McClements)
%	\item H-mode quality diagnostics(div current, magnetics)
\end{enumerate}
\end{frame}

\begin{frame}[allowframebreaks]{Proposed shot plan - AUG}
\vspace{-1cm}
Allocated shots $\sim$ 14 + contingency 
\begin{itemize}
	\item L-Mode (6 shots)
	\begin{enumerate}
		\item Reference shot \# 30276
		\item Perform $I_p$ scan (three values) with fixed $B_T$
		\item Perform $I_p$ scan fixing  $q_{95}$	 ($B_T$)	
		\item $B_T= 2.0$ required by proposal 6 (Aguiam). Should be possible for some shots? 		
		\item should we add OM puff to validate neutral measurements?
		\item Possibility of strike-point sweeping during shot
		\item RFEA measurements in internal program. 
		\item These shots combines experiments proposed by:
                  Carralero (\# 7), Militello (\#1), Vianello (\#13)

	\end{enumerate}
	\item H-mode (9 shots)
	\begin{enumerate}
		\item Reference shot \# 33059 (AUG15-2.2-3). Aim is to achieve conditions
                  in \# 31607 (Sun 6 MW) through careful power and density ramp monitoring new HHF probe 
		\item With clear shoulder formation repeat with midplane probe at varying radial positions 
		\item Strike-point sweep if feasible? 
		\item In all shots, particle acceleration in ELMs will be studied using microwaves,
                  soft X-rays, and FILD 
		\item Reflectometry measurements of density profiles and fluctuations at multiple poloidal locations (Aguiam and Vicente)
		\item These shots combine experiments proposed by:
                  Aguiam (\#6), Carralero (\#7), McClements (\#4),
                  Militello (\# 1, 2), Vianello (\#13), Vicente (\#8)
		
	\end{enumerate}
\end{itemize}
\end{frame}

\begin{frame}{Proposed shot plan - TCV}
	\begin{itemize}
		\item L-mode (11 shots)
		\begin{enumerate}
			\item Reference shot \# 53514
			\item Same $I_p$ and $q_{95}$ scans as on AUG
			\item Two shots with reversed $\bm{B} \times \nabla B$
			\item Move plasma vertically. Disentangle
                          $q_{95}$ and $L_{\|}$	
                        \item These shots combines proposal: Vianello
                          (\#13), Tsui (\#15), Militello (\#1, 2)
		\end{enumerate}
		\item H-mode (12 shots)
		\begin{enumerate}
			\item Reference shot \# 53352
			\item Since H-mode shoulder is new territory on TCV first shots will be scenario development
			\item As on AUG. Incremental increases of power and density. Close monitoring the midplane probe. 
                        \item These shots combines proposal: Vianello
                          (\#13), Tsui (\#15), Militello (\#1, 2)
		\end{enumerate}
	\end{itemize}
\end{frame}


\begin{frame}{Preliminary shot plan  - MAST-U}
	\begin{itemize}
		\item L-mode (6 shots)
		\begin{enumerate}		
			\item New machine. No reference shot yet			
			\item Same $I_p$ and $q_95$ scan to allow
                          cross-machine comparison. Availability of multi pin probe? 
			\item Try varying fueling location
			\item RFEA measurements during shoulder
                          formation 
                        \item These shots combines proposal: Militello
                          (\#1, 2, 3), Walkden (\#11), Vianello (\#13)
		\end{enumerate}
		\item H-mode (7 shots)
		\begin{enumerate}
			\item These experiments require the existence of H-mode reference shot. Perhaps not available in 2017
			\item Similar scenario development as on other emachines
                        \item These shots combines proposal: Militello
                          (\#1, 2, 3),  Vianello (\#13)
		\end{enumerate}
		\item Probe head must be changed to allow
                  investigation of both fast T$_e$ (BPP),  and T$_i$
                  (RFEA) but this require careful schedule of machine time
	\end{itemize}
\end{frame}

\begin{frame}{COMPASS collaboration}
\begin{itemize}
\item COMPASS is not an MST1 device, therefore no operational cost
  will be provided by MST1
\item However GA agrees that MST1 can provide part of the ppy for
  running experiments in line with MST1 topics
\item Proposal \# 14 is focused filaments dynamics and shoulder
  formation both in L and H-Mode
\item Experimental plan can be therefore agreed in order to perform
  similar experiments (e.g. I$_p$ and $q_{95}$ scan in L-Mode)
  ensuring proper comparison between devices
\item COMPASS suppose 3ppy for running and evaluation of experiment
  pertaining Topic-21, among Czeck and other european laboratories.
\end{itemize}
\end{frame}

\begin{frame}{There are urgent experiment which cannot be addressed}
\begin{itemize}
	\item Detailed ion temp measurements. Must eventually be
          completed in internal programs. Are you willing to share the
          information among Topic 21 even though done internally?
	\item Topology investigations? For example the comparison
          USN/LSN/DN is not included. Diagnostic capability of upper
          divertor is weaker than the lower one.
          Better to use reverse B$_t$  which is known
          to have influences in detachment as
          well \parencite{McLean}
	\item Strikepoint sweeping and cryo pump
	\item Parameters scans in H-mode
	\item X-point probes
\end{itemize}
\end{frame}

\begin{frame}{Modeling require}
  \begin{enumerate}
    \item Mean field modeling, in particular for neutral investigation.
      Available for AUG (EMC3-EIRENCE) and
      MAST-U (SOLPS). So far we lack investigation for these scenario
      targeting TCV. Manpower needed from edge and SOL modeling
    \item Turbulence code: we should need self-consistent interaction
      with neutrals. Possible tools \texttt{BOUT ++}, \texttt{HESEL},
      \texttt{GBS}, \texttt{Tokam-3X}.
      Some of them can be run by people who will, very
      likely, apply for this topic, some other not
  \end{enumerate}
\end{frame}

\begin{frame}{Future and Miscellaneous}
	\begin{itemize}
		\item Possibility to piggybag on
                  Quiescent/small ELM H-mode shots (Topic 5,6, 18)
		\item Shortly after this meeting we will organize
                  a follow-up meeting to reconcile ideas
                  from this discussion (date not fixed yet)
		\item All information will be gathered on the wiki
                \item Other channels for discussion/sharing. e.g. Slack
                  (mst1-topic21.slack.com), Github for code sharing, \ldots
		\item We aim to arrive to a proper shared database of
                  filaments between the machines, where properties
                  will be obtained using agreed approaches so that a
                  cross-machine comparison paper could at the end be
                  finalized at a certain point		
	\end{itemize}
\end{frame}

\begin{frame}{Discussion agenda}
	\begin{enumerate}
		\item Priorities. These are our personal ideas. Do we agree? 
		\item \underline{Common measurement techniques}. Filament
                  properties, $\Lambda_{div}$ estimate and location,
                  Neutrals, packing fraction, \ldots 
 	\end{enumerate}
\end{frame}


\end{document}

