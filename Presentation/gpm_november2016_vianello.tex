\documentclass[10pt, compress]{beamer}
\usetheme[conference=MST1-GPM,venue=Garching, date=15/11/2016, titleprogressbar, logo=RFX-logo]{Eurof}
\usepackage{listings,amsmath,multimedia, amssymb}
\usepackage{tangocolors}
\usepackage{rfxcolor}
% for drawing
\usepackage{pgf}
\usepackage{tikz}
\usetikzlibrary{arrows,shapes,backgrounds}
\usepackage{onimage}
\usepackage[export]{adjustbox}
% for font
\usepackage[absolute,overlay]{textpos}
  \setlength{\TPHorizModule}{1mm}
  \setlength{\TPVertModule}{1mm}

\usepackage[style=nature,citestyle=authoryear-comp,defernumbers=true,maxnames=2,firstinits=true,
uniquename=init,backend=bibtex8,arxiv=abs,mcite]{biblatex}
\bibliography{biblio}
\renewcommand*{\bibfont}{\footnotesize}
\renewcommand*{\citesetup}{\footnotesize}
\usepackage[export]{adjustbox}
\makeatother
\mode<presentation>
\makeatletter
% add a macro that saves its argument
\newcommand{\footlineextra}[1]{\gdef\insertfootlineextra{#1}}
\newbox\footlineextrabox
% for reducing font on a single slide
\newcommand\Fontvi{\fontsize{8}{7.2}\selectfont}
\title{Topic 21: Filamentary transport in high-power H-mode conditions and in
  no/small-ELM regimes to predict heat and particle loads on PFCs for
  future devices}
\date{15 November 2016}
\author[N.Vianello]{J. Madsen and N. Vianello}
\begin{document}
\tikzstyle{every picture}+=[remember picture]
\maketitle
\begin{frame}{Present status}
  \setbeamercovered{transparent}
  \begin{enumerate}[<+(1) | invisible@-+>]
    \item Recent experimental and theoretical investigation agrees on
      suggesting SOL density profiles is determined by a balance
      between outward convective filamentary transport and parallel dynamics
    \item The 
     
  \end{itemize}
\end{frame}
% \begin{frame}{Background and objectives}
% \only<1>{
% \begin{itemize}
% \item Role of turbulence transport in the SOL saturation is a well
%   know
%   feature \parencite{LaBombard:2001ks,Rudakov:2005ic,Garcia:2007hh,Carralero:2014gs}. \textcolor{red}{Increasing
%   density, even without reaching detachment SOL profile tend to flatten}
% \end{itemize}}
% \only<2-3>{
%   \begin{columns}
%     \begin{column}{0.5\textwidth}
%       \begin{itemize}
%       \item Present understanding suggests that a transition of filament
%         propagation regime is dominated by effective
%         collisionality $\Lambda$ \parencite{Myra:2006p2754}
%         \begin{equation*}
%           \Lambda = \frac{L_{\parallel}/c_s}{1/\nu_{ei}}\frac{\Omega_i}{\Omega_e}
%         \end{equation*}
%         \item<3> \textcolor{red}{At constant blob size, increasing
%             collisionality provide a change in the filaments velocity
%             scaling properties}
%       \end{itemize}
%     \end{column}
%     \begin{column}{0.5\textwidth}
% %      \includegraphics<2>[width=\textwidth]{../pdfbox/Presentation/FigMyra}
%         \begin{tikzonimage}[width=\textwidth]{../pdfbox/Presentation/FigMyra}
%           \only<3>{
%             \draw [->, ultra thick, red] (0.7,0.3) -- (0.7,0.6);}
%         \end{tikzonimage}
%     \end{column}
%   \end{columns}
% }
% \only<4-6>{
%   \begin{columns}
%     \begin{column}{0.5\textwidth}
%       \begin{itemize}
%       \item AUG and JET \parencite{Carralero:2015gu} suggest that $\Lambda_{div}$ dominates this
%         process and a transition from
%         \textcolor{ta3chameleon}{sheath-limited}
%         \onslide<5-6>{to \textcolor{blue}{inertial regime}} 
%         \item<6> \textcolor{red}{Tested by changing n$_e$
%             and $T_e$ through fueling/seeding/heating}
%       \end{itemize}
%     \end{column}
%     \begin{column}{0.5\textwidth}
%       \begin{tikzonimage}[width=\textwidth]{../pdfbox/Presentation/KoMFig2.png}
%         \only<4-6>{
%           \draw [thick, ta3chameleon, thick] (0.45,0.25) ellipse (0.23 and 0.1);
%       }
%         \only<5-6>{
%           \draw [thick, blue, thick, rotate=-4] (0.8,0.55) ellipse (0.1 and 0.33);
%       }
%         \only<6>{
%           \draw [thick, red, thick, dashed] (0.35,0.75) circle (0.25);
%       }
%     \end{tikzonimage}
%   \end{column}
%   \end{columns}
% }
% \only<7->{
%   \begin{columns}
%     \begin{column}{0.5\textwidth}
%       \begin{itemize}
%       \item MAST \parencite{Militello:2016hk} and old TCV
%         \parencite{Garcia:2007p2615} data 
%         suggested and I$_P$ dependence. 
%           \textcolor{red}{At higher current and
%         same density SOL shoulder disappear} 
%       \item<8> \textcolor{ta3chameleon}{Being shoulder formation
%           induced by a change of $\Gamma_{\perp}$ \textit{vs}
%           $\Gamma_{\parallel}$ balance we will try to test dependence
%           on L$_{\parallel}$}
%     \end{itemize}
%     \end{column}
%     \begin{column}{0.5\textwidth}
%       \includegraphics[width=\textwidth]{../pdfbox/Presentation/FigMilitello.pdf}
%     \end{column}
%   \end{columns}
% }
% \centering{\includegraphics<1>[width=1.1\textwidth]{../pdfbox/Presentation/Fig1Presentation}}
% \end{frame}

% \begin{frame}{Flux expansion scan/1}
% \vspace{-1cm}
%   \Fontvi
%   % \begin{columns}
%   %   \begin{column}{0.3\textwidth}
%       \centering{\includegraphics[width=.75\textwidth]{../pdfbox/Fig1a}}
%       \begin{itemize}
%       \item Parallel connection length varied by scanning 
%         flux expansion during density
%         ramps in ohmically heated discharges. \textcolor{red}{$\nabla B_i$ towards the X-point}
%         %\begin{equation*}
%         \[  f_x = \frac{(B_p/B_t)_{MP}}{(B_p/B_t)_{SP}}\]
%         %\end{equation*}
%         \item Variation of $f_x$ change the volume of flux tube as
%           well as parallel connection length (Increased up to 70\%)
%       \end{itemize}
%     % \end{column}
%     % \begin{column}{0.7\textwidth}
%   %   \end{column}
%   % \end{columns}
% \end{frame}

% \begin{frame}{Flux expansion scan/2}
%   \Fontvi
%       \begin{itemize}
%       \item<1-> Flux expansion variation induce modification at the target
%       \item<1-> Radiation front from bolometry  moves earlier in
%         density at larger f$_x$. Front movement trails CIII cut-off
%         movement (Reimerdes IAEA 2016)
%       \item<2-> Volumetric recombination rates from $n=6, 7$ balmer line
%         emission \parencite{kevin:jnm} indicates that \textcolor{red}{recombination occurs
%           earlier in density at high $f_x$}
%       \item<3|only@3> \textcolor{ta3skyblue}{This is not confirmed in reversed B$_t$ where
%         increasing the f$_x$ lead to deeper detachment without}
%         changing density threshold  
%     \end{itemize}
%     \only<1-2>{\begin{tikzonimage}[width=\textwidth]{../pdfbox/Fig2c}
%       \only<1>{\fill[white] (0.,0) rectangle (0.52,1); }    
%       \draw [->, ultra thick, black] (0.79,0.69) -- (0.75,0.51);
%     \end{tikzonimage}}
% \centering{\includegraphics<3>[width=.5\textwidth]{../pdfbox/Presentation/XD_detach_HM}}
%   \end{frame}

% \begin{frame}{Flux expansion scan/3}
%   \vspace{-1cm}
% \Fontvi
%       \begin{itemize}
%       \item Target and upstream profile measured at the same density
%         but different f$_x$
%       \item<3-> Target profiles in the near SOL modified with broader profile at higher
%         f$_x$. \textcolor{red}{Huge variation on $\Lambda_{div}$
%           without upstream modification}
%       \item<4|only@4> \textcolor{ta3chameleon}{Only further increase
%           of density (although $\Lambda_{div}$ in the far SOL
%           decreases) leads to profile broadening}
%       \end{itemize}
%       \centering{\includegraphics<1>[width=.65\textwidth]{../pdfbox/Fig4Series_1}}
%       \centering{\includegraphics<2>[width=.65\textwidth]{../pdfbox/Fig4Series_2}}
%       \centering{\includegraphics<3>[width=.65\textwidth]{../pdfbox/Fig4Series_3}}
%       \centering{\includegraphics<4>[width=.65\textwidth]{../pdfbox/Fig4Series_4}}
% \end{frame}

% \begin{frame}{Filaments effect}
%   \vspace{-1cm}
% \Fontvi
%       \begin{itemize}
%       \item<1-3> Blob properties deduced from fast reciprocating as done in \parencite{Boedo:2001tt}
%       \item<1|only@1> Blob size deduced as $\delta_b = \tau_b v_{\perp}$
%         % \begin{equation*}
%         %   \tau_b = \text{FWHM of CAS} \qquad v_{\perp} = \sqrt{v_{E\times B, r}^2 +
%         %     v_{E\times B, p}^2}
%         % \end{equation*}
%       \item<2-3> No clear dependence on
%           $\Lambda_{div}$ during f$_x$ scan
%       \item<3> \textcolor{ta3chameleon}{When we increase the
%           density filaments are bigger at same $\Lambda_{div}$}
%       \item<4-> \textcolor{ta3skyblue}{On a statistical basis,
%           blob size increases with density without clear dependence on
%           f$_x$ \onslide<5>{\textcolor{red}{as well as $\lambda_n$ in the Far SOL}}}
%       \item<6> \textcolor{ta3chameleon}{Behavior at very high
%           density to be confirmed}  
%       \end{itemize}
%       \centering{\includegraphics<1>[width=.53\textwidth]{../pdfbox/Fig14b}}
%       \centering{\includegraphics<2>[width=.63\textwidth]{../pdfbox/Fig7}}
%       \centering{\includegraphics<3>[width=.63\textwidth]{../pdfbox/Fig7b}}
%       \centering{\includegraphics<4>[width=.85\textwidth]{../pdfbox/Fig12}}
%       \centering{\includegraphics<5>[width=.85\textwidth]{../pdfbox/Fig12b}}
%       \only<6>{
%         \begin{center}
%           \begin{tikzonimage}[width=.85\textwidth]{../pdfbox/Fig12b}
%           \draw [thick, ta3chameleon, thick] (0.75,0.5) ellipse (0.1 and 0.1);
%           \end{tikzonimage}
%         \end{center}
%       }
% \end{frame}

% \begin{frame}{Current scan/1}
%   \begin{columns}
%     \begin{column}{0.5\textwidth}
%       \begin{itemize}
%       \item<1-3> On a single shape (f$_x \approx 4$) current scan with
%         similar density ramps performed
%       \item<2-3> \textcolor{red}{At lower current indication of shoulder for $R-R_{sep}
%         \geq 1$ cm which disappear at higher current (same density)}
%       \item<3> \textcolor{ta3chameleon}{At even higher current we need to substantially
%         increase the fueling (even if at the same $n/n_G$)}
%       \end{itemize}
%     \end{column}
%     \begin{column}{0.5\textwidth}
%       \only<1-2>{
%       \begin{tikzonimage}[width=\textwidth]{../pdfbox/Fig4d_series1}
%         \only<2>{
%           \draw [thick, red, thick] (0.75,0.75) ellipse (0.2 and 0.1);
%         }
%       \end{tikzonimage} }
%     \only<3>{
%       \begin{tikzonimage}[width=\textwidth]{../pdfbox/Fig4d_series2}
%           \draw [thick, ta3chameleon, thick] (0.67,0.35) ellipse (0.22 and 0.1);
%           \draw [thick, ta3chameleon, thick] (0.5,0.14) ellipse (0.3 and 0.03);
%         \end{tikzonimage}      
%       }
%     \end{column}
%   \end{columns}
% \end{frame}

% \begin{frame}{Current scan/2}
%   \begin{columns}
%     \begin{column}{0.5\textwidth}
%       \begin{itemize}
%       \item On a statistical basis we have a slight
%         reduction of radial velocity of the filaments as in MAST \parencite{Kirk:2016jj} with unclear
%         effect on blob-size
%       \end{itemize}
%     \end{column}
%       \begin{column}{0.5\textwidth}
%      \begin{tikzonimage}[width=\textwidth]{../pdfbox/Fig8}
%        \draw [->, ultra thick, red, dashed] (0.33,0.37) -- (0.7,0.27);
%      \end{tikzonimage}
%    \end{column}
%   \end{columns}
%   \end{frame}

% \begin{frame}{Lower Single Null (LSN) and Double Null (DN)}
%   \begin{columns}
%     \begin{column}{0.5\textwidth}
%       \begin{itemize}
%       \item<1|only@1> We compare LSN \textit{vs} DN on a single discharge at
%         two levels of density
%         \item<2|only@2> At lower density values the radiation is spread among
%           upper and lower X-points, whereas at higher density radiation
%           is higher at lower X-point
%         \item<6|only@6> Blob size increases with density with DN cases always
%           exhibiting larger blobs. Still compatible with general density
%           scaling. \textcolor{orange}{Radial velocity seems to decrease for DN}
%       \end{itemize}
%     \end{column}
%     \begin{column}{0.5\textwidth}
%       \includegraphics<1>[trim={0 7.6cm 0 0}, clip, width=\textwidth]{../pdfbox/Fig5}
%       \includegraphics<2>[width=\textwidth]{../pdfbox/Fig5c}
%       \only<6>{
%         \begin{tikzonimage}[width=\textwidth]{../pdfbox/Fig9}
%           \draw [->, thick, orange] (0.33, 0.33) -- (0.83, 0.21);
%         \end{tikzonimage}
%       }
%     \end{column}
%   \end{columns}
%   \only<3>{
%     \begin{itemize}
%       \item At the LFS target, DN density profile at lower level of
%         density is reduced and closer to 
%         what observed at higher density
%       \end{itemize}
%       \begin{tikzonimage}[width=\textwidth]{../pdfbox/Fig5e}
%         \draw [dashed, red, ultra thick] (0.7, 0.37) -- (0.3, 0.37);
%         \draw [->, dashed, red, ultra thick] (0.3, 0.7) -- (0.3, 0.38);
%       \end{tikzonimage}
%     }
%   \only<4-5>{
%     \begin{itemize}
%       \item Upstream profile suggests that a stronger tendency of
%         developing density shoulder is observed in DN 
%       \end{itemize}
%       \begin{tikzonimage}[width=\textwidth]{../pdfbox/Fig5d}
%         \only<5>{\draw[dashed, red, thick] (1.0, 0.55) -- (0.1, 0.55);}
%       \end{tikzonimage}
%     }

% \end{frame}

% \begin{frame}{Upper, Lower and Double Null}
%     \Fontvi
%     \begin{itemize}
%       \item The same density ramp performed in Upper, Lower and Double
%         null with all the Strikes points at the inner wall
%       \item This allow to test any possible dependence from $\nabla
%         B_i$ drift direction. C-Mod suggests difference in the medium
%         density range \parencite{LaBombard:2004kg}
%     \end{itemize}
%     \centering{\includegraphics[width=0.9\textwidth]{../pdfbox/Fig1b}}
% \end{frame}
% \begin{frame}{Upper, Lower and Double Null}
%     \Fontvi
%     \begin{itemize}
%       \item<2-> Lower and Double null blob size scale consistently with
%         density scaling
%       \item<3->  Upper Single null always exhibits
%         larger blob
%       \end{itemize}
%       \vspace{-0.27cm}
%     \centering{\includegraphics<1>[width=.6\textwidth]{../pdfbox/Fig11b_Series1}}
%     \centering{\includegraphics<2>[width=.6\textwidth]{../pdfbox/Fig11b_Series2}}
%     \centering{\includegraphics<3>[width=.6\textwidth]{../pdfbox/Fig11b_Series3}}
% \end{frame}  

% \begin{frame}{Summary}
%   \Fontvi
%   \begin{itemize}
%   \item<1-> $\lambda_n = |\nabla n_e/n_e|^{-1}$ computed around 1cm from the
%     separatrix
%   \item<2-> $\lambda_n$ clearly increases with $\Theta = (\delta_b
%         R^{1/5}/L_{\parallel}^{2/5}\rho_s^{4/5})^{5/2}$,
%     i.e. with blob-size
%   \item<3-> \alert{The dependence on L$_{\parallel}$ is marginal
%       although we obtain 
%       larger $\Theta$ at smaller $L_{\parallel}$}.
%   \item<4-> The increase of $\lambda_n$ is consistent with an increase
%     of perpendicular to parallel losses
%   \item<5-> The dependence on $\Lambda_{div}$ is weaker than in
%     AUG/JET. \onslide<6>{\textcolor{ta3chameleon}{ Large $\Lambda$ is not
%         sufficient for flat profile.}}
      
%   \end{itemize}
%   \begin{tikzonimage}[width=\textwidth]{../pdfbox/Fig10c}
%     \only<1>{\fill[white] (0, 0) rectangle (1, 1);}
%     \only<2>{\fill[white] (0.348,0) rectangle (1,1); }    
%     \only<3>{\fill[white] (0.348,0) rectangle (0.897,1); }    
%     \only<4>{\fill[white] (0.62,0) rectangle (0.897,1); }    
%     \only<6>{
%       \draw [thick, ta3chameleon, thick, rotate = 4] (0.865,0.2) ellipse (0.04 and 0.2);
%       \draw [thick, ta3chameleon, thick, rotate = 3] (0.72,0.6) ellipse (0.03 and 0.15);
%  }    
%     \end{tikzonimage}
% \end{frame}

\end{document}
%\draw [thick, blue, thick, rotate=-4] (0.8,0.55) ellipse (0.1 and 0.33);
